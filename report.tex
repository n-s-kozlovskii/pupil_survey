
% Default to the notebook output style

    


% Inherit from the specified cell style.




    
\documentclass[11pt]{article}

    
    
    \usepackage[T1]{fontenc}
    % Nicer default font (+ math font) than Computer Modern for most use cases
    \usepackage{mathpazo}

    % Basic figure setup, for now with no caption control since it's done
    % automatically by Pandoc (which extracts ![](path) syntax from Markdown).
    \usepackage{graphicx}
    % We will generate all images so they have a width \maxwidth. This means
    % that they will get their normal width if they fit onto the page, but
    % are scaled down if they would overflow the margins.
    \makeatletter
    \def\maxwidth{\ifdim\Gin@nat@width>\linewidth\linewidth
    \else\Gin@nat@width\fi}
    \makeatother
    \let\Oldincludegraphics\includegraphics
    % Set max figure width to be 80% of text width, for now hardcoded.
    \renewcommand{\includegraphics}[1]{\Oldincludegraphics[width=.8\maxwidth]{#1}}
    % Ensure that by default, figures have no caption (until we provide a
    % proper Figure object with a Caption API and a way to capture that
    % in the conversion process - todo).
    \usepackage{caption}
    \DeclareCaptionLabelFormat{nolabel}{}
    \captionsetup{labelformat=nolabel}

    \usepackage{adjustbox} % Used to constrain images to a maximum size 
    \usepackage{xcolor} % Allow colors to be defined
    \usepackage{enumerate} % Needed for markdown enumerations to work
    \usepackage{geometry} % Used to adjust the document margins
    \usepackage{amsmath} % Equations
    \usepackage{amssymb} % Equations
    \usepackage{textcomp} % defines textquotesingle
    % Hack from http://tex.stackexchange.com/a/47451/13684:
    \AtBeginDocument{%
        \def\PYZsq{\textquotesingle}% Upright quotes in Pygmentized code
    }
    \usepackage{upquote} % Upright quotes for verbatim code
    \usepackage{eurosym} % defines \euro
    \usepackage[mathletters]{ucs} % Extended unicode (utf-8) support
    \usepackage[utf8x]{inputenc} % Allow utf-8 characters in the tex document
    \usepackage{fancyvrb} % verbatim replacement that allows latex
    \usepackage{grffile} % extends the file name processing of package graphics 
                         % to support a larger range 
    % The hyperref package gives us a pdf with properly built
    % internal navigation ('pdf bookmarks' for the table of contents,
    % internal cross-reference links, web links for URLs, etc.)
    \usepackage{hyperref}
    \usepackage{longtable} % longtable support required by pandoc >1.10
    \usepackage{booktabs}  % table support for pandoc > 1.12.2
    \usepackage[inline]{enumitem} % IRkernel/repr support (it uses the enumerate* environment)
    \usepackage[normalem]{ulem} % ulem is needed to support strikethroughs (\sout)
                                % normalem makes italics be italics, not underlines
    
	\usepackage[russian]{babel}
    
    
    % Colors for the hyperref package
    \definecolor{urlcolor}{rgb}{0,.145,.698}
    \definecolor{linkcolor}{rgb}{.71,0.21,0.01}
    \definecolor{citecolor}{rgb}{.12,.54,.11}

    % ANSI colors
    \definecolor{ansi-black}{HTML}{3E424D}
    \definecolor{ansi-black-intense}{HTML}{282C36}
    \definecolor{ansi-red}{HTML}{E75C58}
    \definecolor{ansi-red-intense}{HTML}{B22B31}
    \definecolor{ansi-green}{HTML}{00A250}
    \definecolor{ansi-green-intense}{HTML}{007427}
    \definecolor{ansi-yellow}{HTML}{DDB62B}
    \definecolor{ansi-yellow-intense}{HTML}{B27D12}
    \definecolor{ansi-blue}{HTML}{208FFB}
    \definecolor{ansi-blue-intense}{HTML}{0065CA}
    \definecolor{ansi-magenta}{HTML}{D160C4}
    \definecolor{ansi-magenta-intense}{HTML}{A03196}
    \definecolor{ansi-cyan}{HTML}{60C6C8}
    \definecolor{ansi-cyan-intense}{HTML}{258F8F}
    \definecolor{ansi-white}{HTML}{C5C1B4}
    \definecolor{ansi-white-intense}{HTML}{A1A6B2}

    % commands and environments needed by pandoc snippets
    % extracted from the output of `pandoc -s`
    \providecommand{\tightlist}{%
      \setlength{\itemsep}{0pt}\setlength{\parskip}{0pt}}
    \DefineVerbatimEnvironment{Highlighting}{Verbatim}{commandchars=\\\{\}}
    % Add ',fontsize=\small' for more characters per line
    \newenvironment{Shaded}{}{}
    \newcommand{\KeywordTok}[1]{\textcolor[rgb]{0.00,0.44,0.13}{\textbf{{#1}}}}
    \newcommand{\DataTypeTok}[1]{\textcolor[rgb]{0.56,0.13,0.00}{{#1}}}
    \newcommand{\DecValTok}[1]{\textcolor[rgb]{0.25,0.63,0.44}{{#1}}}
    \newcommand{\BaseNTok}[1]{\textcolor[rgb]{0.25,0.63,0.44}{{#1}}}
    \newcommand{\FloatTok}[1]{\textcolor[rgb]{0.25,0.63,0.44}{{#1}}}
    \newcommand{\CharTok}[1]{\textcolor[rgb]{0.25,0.44,0.63}{{#1}}}
    \newcommand{\StringTok}[1]{\textcolor[rgb]{0.25,0.44,0.63}{{#1}}}
    \newcommand{\CommentTok}[1]{\textcolor[rgb]{0.38,0.63,0.69}{\textit{{#1}}}}
    \newcommand{\OtherTok}[1]{\textcolor[rgb]{0.00,0.44,0.13}{{#1}}}
    \newcommand{\AlertTok}[1]{\textcolor[rgb]{1.00,0.00,0.00}{\textbf{{#1}}}}
    \newcommand{\FunctionTok}[1]{\textcolor[rgb]{0.02,0.16,0.49}{{#1}}}
    \newcommand{\RegionMarkerTok}[1]{{#1}}
    \newcommand{\ErrorTok}[1]{\textcolor[rgb]{1.00,0.00,0.00}{\textbf{{#1}}}}
    \newcommand{\NormalTok}[1]{{#1}}
    
    % Additional commands for more recent versions of Pandoc
    \newcommand{\ConstantTok}[1]{\textcolor[rgb]{0.53,0.00,0.00}{{#1}}}
    \newcommand{\SpecialCharTok}[1]{\textcolor[rgb]{0.25,0.44,0.63}{{#1}}}
    \newcommand{\VerbatimStringTok}[1]{\textcolor[rgb]{0.25,0.44,0.63}{{#1}}}
    \newcommand{\SpecialStringTok}[1]{\textcolor[rgb]{0.73,0.40,0.53}{{#1}}}
    \newcommand{\ImportTok}[1]{{#1}}
    \newcommand{\DocumentationTok}[1]{\textcolor[rgb]{0.73,0.13,0.13}{\textit{{#1}}}}
    \newcommand{\AnnotationTok}[1]{\textcolor[rgb]{0.38,0.63,0.69}{\textbf{\textit{{#1}}}}}
    \newcommand{\CommentVarTok}[1]{\textcolor[rgb]{0.38,0.63,0.69}{\textbf{\textit{{#1}}}}}
    \newcommand{\VariableTok}[1]{\textcolor[rgb]{0.10,0.09,0.49}{{#1}}}
    \newcommand{\ControlFlowTok}[1]{\textcolor[rgb]{0.00,0.44,0.13}{\textbf{{#1}}}}
    \newcommand{\OperatorTok}[1]{\textcolor[rgb]{0.40,0.40,0.40}{{#1}}}
    \newcommand{\BuiltInTok}[1]{{#1}}
    \newcommand{\ExtensionTok}[1]{{#1}}
    \newcommand{\PreprocessorTok}[1]{\textcolor[rgb]{0.74,0.48,0.00}{{#1}}}
    \newcommand{\AttributeTok}[1]{\textcolor[rgb]{0.49,0.56,0.16}{{#1}}}
    \newcommand{\InformationTok}[1]{\textcolor[rgb]{0.38,0.63,0.69}{\textbf{\textit{{#1}}}}}
    \newcommand{\WarningTok}[1]{\textcolor[rgb]{0.38,0.63,0.69}{\textbf{\textit{{#1}}}}}
    
    
    % Define a nice break command that doesn't care if a line doesn't already
    % exist.
    \def\br{\hspace*{\fill} \\* }
    % Math Jax compatability definitions
    \def\gt{>}
    \def\lt{<}
    % Document parameters
    \title{pupil\_survey}
    
    
    

    % Pygments definitions
    
\makeatletter
\def\PY@reset{\let\PY@it=\relax \let\PY@bf=\relax%
    \let\PY@ul=\relax \let\PY@tc=\relax%
    \let\PY@bc=\relax \let\PY@ff=\relax}
\def\PY@tok#1{\csname PY@tok@#1\endcsname}
\def\PY@toks#1+{\ifx\relax#1\empty\else%
    \PY@tok{#1}\expandafter\PY@toks\fi}
\def\PY@do#1{\PY@bc{\PY@tc{\PY@ul{%
    \PY@it{\PY@bf{\PY@ff{#1}}}}}}}
\def\PY#1#2{\PY@reset\PY@toks#1+\relax+\PY@do{#2}}

\expandafter\def\csname PY@tok@gs\endcsname{\let\PY@bf=\textbf}
\expandafter\def\csname PY@tok@kn\endcsname{\let\PY@bf=\textbf\def\PY@tc##1{\textcolor[rgb]{0.00,0.50,0.00}{##1}}}
\expandafter\def\csname PY@tok@err\endcsname{\def\PY@bc##1{\setlength{\fboxsep}{0pt}\fcolorbox[rgb]{1.00,0.00,0.00}{1,1,1}{\strut ##1}}}
\expandafter\def\csname PY@tok@kr\endcsname{\let\PY@bf=\textbf\def\PY@tc##1{\textcolor[rgb]{0.00,0.50,0.00}{##1}}}
\expandafter\def\csname PY@tok@cs\endcsname{\let\PY@it=\textit\def\PY@tc##1{\textcolor[rgb]{0.25,0.50,0.50}{##1}}}
\expandafter\def\csname PY@tok@gt\endcsname{\def\PY@tc##1{\textcolor[rgb]{0.00,0.27,0.87}{##1}}}
\expandafter\def\csname PY@tok@kd\endcsname{\let\PY@bf=\textbf\def\PY@tc##1{\textcolor[rgb]{0.00,0.50,0.00}{##1}}}
\expandafter\def\csname PY@tok@m\endcsname{\def\PY@tc##1{\textcolor[rgb]{0.40,0.40,0.40}{##1}}}
\expandafter\def\csname PY@tok@nb\endcsname{\def\PY@tc##1{\textcolor[rgb]{0.00,0.50,0.00}{##1}}}
\expandafter\def\csname PY@tok@k\endcsname{\let\PY@bf=\textbf\def\PY@tc##1{\textcolor[rgb]{0.00,0.50,0.00}{##1}}}
\expandafter\def\csname PY@tok@no\endcsname{\def\PY@tc##1{\textcolor[rgb]{0.53,0.00,0.00}{##1}}}
\expandafter\def\csname PY@tok@kt\endcsname{\def\PY@tc##1{\textcolor[rgb]{0.69,0.00,0.25}{##1}}}
\expandafter\def\csname PY@tok@ch\endcsname{\let\PY@it=\textit\def\PY@tc##1{\textcolor[rgb]{0.25,0.50,0.50}{##1}}}
\expandafter\def\csname PY@tok@gh\endcsname{\let\PY@bf=\textbf\def\PY@tc##1{\textcolor[rgb]{0.00,0.00,0.50}{##1}}}
\expandafter\def\csname PY@tok@s\endcsname{\def\PY@tc##1{\textcolor[rgb]{0.73,0.13,0.13}{##1}}}
\expandafter\def\csname PY@tok@sx\endcsname{\def\PY@tc##1{\textcolor[rgb]{0.00,0.50,0.00}{##1}}}
\expandafter\def\csname PY@tok@sb\endcsname{\def\PY@tc##1{\textcolor[rgb]{0.73,0.13,0.13}{##1}}}
\expandafter\def\csname PY@tok@nn\endcsname{\let\PY@bf=\textbf\def\PY@tc##1{\textcolor[rgb]{0.00,0.00,1.00}{##1}}}
\expandafter\def\csname PY@tok@gu\endcsname{\let\PY@bf=\textbf\def\PY@tc##1{\textcolor[rgb]{0.50,0.00,0.50}{##1}}}
\expandafter\def\csname PY@tok@mb\endcsname{\def\PY@tc##1{\textcolor[rgb]{0.40,0.40,0.40}{##1}}}
\expandafter\def\csname PY@tok@bp\endcsname{\def\PY@tc##1{\textcolor[rgb]{0.00,0.50,0.00}{##1}}}
\expandafter\def\csname PY@tok@c\endcsname{\let\PY@it=\textit\def\PY@tc##1{\textcolor[rgb]{0.25,0.50,0.50}{##1}}}
\expandafter\def\csname PY@tok@dl\endcsname{\def\PY@tc##1{\textcolor[rgb]{0.73,0.13,0.13}{##1}}}
\expandafter\def\csname PY@tok@vm\endcsname{\def\PY@tc##1{\textcolor[rgb]{0.10,0.09,0.49}{##1}}}
\expandafter\def\csname PY@tok@sh\endcsname{\def\PY@tc##1{\textcolor[rgb]{0.73,0.13,0.13}{##1}}}
\expandafter\def\csname PY@tok@sd\endcsname{\let\PY@it=\textit\def\PY@tc##1{\textcolor[rgb]{0.73,0.13,0.13}{##1}}}
\expandafter\def\csname PY@tok@ss\endcsname{\def\PY@tc##1{\textcolor[rgb]{0.10,0.09,0.49}{##1}}}
\expandafter\def\csname PY@tok@ne\endcsname{\let\PY@bf=\textbf\def\PY@tc##1{\textcolor[rgb]{0.82,0.25,0.23}{##1}}}
\expandafter\def\csname PY@tok@sa\endcsname{\def\PY@tc##1{\textcolor[rgb]{0.73,0.13,0.13}{##1}}}
\expandafter\def\csname PY@tok@sc\endcsname{\def\PY@tc##1{\textcolor[rgb]{0.73,0.13,0.13}{##1}}}
\expandafter\def\csname PY@tok@gd\endcsname{\def\PY@tc##1{\textcolor[rgb]{0.63,0.00,0.00}{##1}}}
\expandafter\def\csname PY@tok@kp\endcsname{\def\PY@tc##1{\textcolor[rgb]{0.00,0.50,0.00}{##1}}}
\expandafter\def\csname PY@tok@gp\endcsname{\let\PY@bf=\textbf\def\PY@tc##1{\textcolor[rgb]{0.00,0.00,0.50}{##1}}}
\expandafter\def\csname PY@tok@s1\endcsname{\def\PY@tc##1{\textcolor[rgb]{0.73,0.13,0.13}{##1}}}
\expandafter\def\csname PY@tok@s2\endcsname{\def\PY@tc##1{\textcolor[rgb]{0.73,0.13,0.13}{##1}}}
\expandafter\def\csname PY@tok@cm\endcsname{\let\PY@it=\textit\def\PY@tc##1{\textcolor[rgb]{0.25,0.50,0.50}{##1}}}
\expandafter\def\csname PY@tok@vc\endcsname{\def\PY@tc##1{\textcolor[rgb]{0.10,0.09,0.49}{##1}}}
\expandafter\def\csname PY@tok@ni\endcsname{\let\PY@bf=\textbf\def\PY@tc##1{\textcolor[rgb]{0.60,0.60,0.60}{##1}}}
\expandafter\def\csname PY@tok@w\endcsname{\def\PY@tc##1{\textcolor[rgb]{0.73,0.73,0.73}{##1}}}
\expandafter\def\csname PY@tok@se\endcsname{\let\PY@bf=\textbf\def\PY@tc##1{\textcolor[rgb]{0.73,0.40,0.13}{##1}}}
\expandafter\def\csname PY@tok@nd\endcsname{\def\PY@tc##1{\textcolor[rgb]{0.67,0.13,1.00}{##1}}}
\expandafter\def\csname PY@tok@nc\endcsname{\let\PY@bf=\textbf\def\PY@tc##1{\textcolor[rgb]{0.00,0.00,1.00}{##1}}}
\expandafter\def\csname PY@tok@kc\endcsname{\let\PY@bf=\textbf\def\PY@tc##1{\textcolor[rgb]{0.00,0.50,0.00}{##1}}}
\expandafter\def\csname PY@tok@ge\endcsname{\let\PY@it=\textit}
\expandafter\def\csname PY@tok@vg\endcsname{\def\PY@tc##1{\textcolor[rgb]{0.10,0.09,0.49}{##1}}}
\expandafter\def\csname PY@tok@nt\endcsname{\let\PY@bf=\textbf\def\PY@tc##1{\textcolor[rgb]{0.00,0.50,0.00}{##1}}}
\expandafter\def\csname PY@tok@si\endcsname{\let\PY@bf=\textbf\def\PY@tc##1{\textcolor[rgb]{0.73,0.40,0.53}{##1}}}
\expandafter\def\csname PY@tok@mi\endcsname{\def\PY@tc##1{\textcolor[rgb]{0.40,0.40,0.40}{##1}}}
\expandafter\def\csname PY@tok@c1\endcsname{\let\PY@it=\textit\def\PY@tc##1{\textcolor[rgb]{0.25,0.50,0.50}{##1}}}
\expandafter\def\csname PY@tok@cpf\endcsname{\let\PY@it=\textit\def\PY@tc##1{\textcolor[rgb]{0.25,0.50,0.50}{##1}}}
\expandafter\def\csname PY@tok@o\endcsname{\def\PY@tc##1{\textcolor[rgb]{0.40,0.40,0.40}{##1}}}
\expandafter\def\csname PY@tok@fm\endcsname{\def\PY@tc##1{\textcolor[rgb]{0.00,0.00,1.00}{##1}}}
\expandafter\def\csname PY@tok@nv\endcsname{\def\PY@tc##1{\textcolor[rgb]{0.10,0.09,0.49}{##1}}}
\expandafter\def\csname PY@tok@nf\endcsname{\def\PY@tc##1{\textcolor[rgb]{0.00,0.00,1.00}{##1}}}
\expandafter\def\csname PY@tok@mf\endcsname{\def\PY@tc##1{\textcolor[rgb]{0.40,0.40,0.40}{##1}}}
\expandafter\def\csname PY@tok@gi\endcsname{\def\PY@tc##1{\textcolor[rgb]{0.00,0.63,0.00}{##1}}}
\expandafter\def\csname PY@tok@mo\endcsname{\def\PY@tc##1{\textcolor[rgb]{0.40,0.40,0.40}{##1}}}
\expandafter\def\csname PY@tok@il\endcsname{\def\PY@tc##1{\textcolor[rgb]{0.40,0.40,0.40}{##1}}}
\expandafter\def\csname PY@tok@mh\endcsname{\def\PY@tc##1{\textcolor[rgb]{0.40,0.40,0.40}{##1}}}
\expandafter\def\csname PY@tok@gr\endcsname{\def\PY@tc##1{\textcolor[rgb]{1.00,0.00,0.00}{##1}}}
\expandafter\def\csname PY@tok@nl\endcsname{\def\PY@tc##1{\textcolor[rgb]{0.63,0.63,0.00}{##1}}}
\expandafter\def\csname PY@tok@ow\endcsname{\let\PY@bf=\textbf\def\PY@tc##1{\textcolor[rgb]{0.67,0.13,1.00}{##1}}}
\expandafter\def\csname PY@tok@go\endcsname{\def\PY@tc##1{\textcolor[rgb]{0.53,0.53,0.53}{##1}}}
\expandafter\def\csname PY@tok@sr\endcsname{\def\PY@tc##1{\textcolor[rgb]{0.73,0.40,0.53}{##1}}}
\expandafter\def\csname PY@tok@cp\endcsname{\def\PY@tc##1{\textcolor[rgb]{0.74,0.48,0.00}{##1}}}
\expandafter\def\csname PY@tok@na\endcsname{\def\PY@tc##1{\textcolor[rgb]{0.49,0.56,0.16}{##1}}}
\expandafter\def\csname PY@tok@vi\endcsname{\def\PY@tc##1{\textcolor[rgb]{0.10,0.09,0.49}{##1}}}

\def\PYZbs{\char`\\}
\def\PYZus{\char`\_}
\def\PYZob{\char`\{}
\def\PYZcb{\char`\}}
\def\PYZca{\char`\^}
\def\PYZam{\char`\&}
\def\PYZlt{\char`\<}
\def\PYZgt{\char`\>}
\def\PYZsh{\char`\#}
\def\PYZpc{\char`\%}
\def\PYZdl{\char`\$}
\def\PYZhy{\char`\-}
\def\PYZsq{\char`\'}
\def\PYZdq{\char`\"}
\def\PYZti{\char`\~}
% for compatibility with earlier versions
\def\PYZat{@}
\def\PYZlb{[}
\def\PYZrb{]}
\makeatother


    % Exact colors from NB
    \definecolor{incolor}{rgb}{0.0, 0.0, 0.5}
    \definecolor{outcolor}{rgb}{0.545, 0.0, 0.0}



    
    % Prevent overflowing lines due to hard-to-break entities
    \sloppy 
    % Setup hyperref package
    \hypersetup{
      breaklinks=true,  % so long urls are correctly broken across lines
      colorlinks=true,
      urlcolor=urlcolor,
      linkcolor=linkcolor,
      citecolor=citecolor,
      }
    % Slightly bigger margins than the latex defaults
    
    \geometry{verbose,tmargin=1in,bmargin=1in,lmargin=1in,rmargin=1in}
    
    

    \begin{document}
    
    
    \title{Анализ опроса словацкой молодежи 2013}
    \author
{Никита Козловский\\
\\
\normalsize{ВГУ, ПММ, 3 группа}\\}
\date{}
\maketitle
\thispagestyle{empty}
\newpage
\tableofcontents
\thispagestyle{empty}
\newpage

\section{Введение}
    Рассмотрим данные, представленные на сайте \url{https://www.kaggle.com/miroslavsabo/young-people-survey}
	Опрос был проведен факультетом факультет социальных и экономических наук Университета имени Коменского в Братиславе в 2013 году. Было опрошено 1010 человек по 150 пунктам: 
	\begin{itemize}
\item Предпочтения в музыке (19 пунктов)
\item Предпочтения в фильмах (12 пунктов)
\item Хобби и интересы (32 пунктов)
\item Фобии (10 пунктов)
\item Здравоохранение (3 пунктов)
\item Черты характера, взгляды на жизнь и мнения (57 пунктов)
\item На что тратите деньги (7 пунктов)
\item Демографические данные (10 пунктов)
\end{itemize}

Попробуем найти ответы на следующие вопросы:
	\begin{itemize}
\item Можно ли разделить студентов на группы по предпочтениям в музыке? (описательные, факторный )
\item Что определяет бережливого человека ? (регрессия) 
\item Анализ страхов по группам 
\end{itemize}

Пример данных

  \begin{Verbatim}[commandchars=\\\{\}]
{\color{incolor}In [{\color{incolor}24}]:} \PY{n}{df}\PY{o}{.}\PY{n}{head}\PY{p}{(}\PY{l+m+mi}{4}\PY{p}{)}
\end{Verbatim}


\begin{Verbatim}[commandchars=\\\{\}]
{\color{outcolor}Out[{\color{outcolor}24}]:}    Music  Slow songs or fast songs  Dance  Folk  Country  Classical music  \textbackslash{}
         0    5.0                       3.0    2.0   1.0      2.0              2.0   
         1    4.0                       4.0    2.0   1.0      1.0              1.0   
         2    5.0                       5.0    2.0   2.0      3.0              4.0   
         3    5.0                       3.0    2.0   1.0      1.0              1.0   
         
            Musical  Pop  Rock  Metal or Hardrock           {\ldots}             Age  \textbackslash{}
         0      1.0  5.0   5.0                1.0           {\ldots}            20.0   
         1      2.0  3.0   5.0                4.0           {\ldots}            19.0   
         2      5.0  3.0   5.0                3.0           {\ldots}            20.0   
         3      1.0  2.0   2.0                1.0           {\ldots}            22.0   
         
            Height  Weight  Number of siblings  Gender  Left - right handed  \textbackslash{}
         0   163.0    48.0                 1.0  female         right handed   
         1   163.0    58.0                 2.0  female         right handed   
         2   176.0    67.0                 2.0  female         right handed   
         3   172.0    59.0                 1.0  female         right handed   
         
                          Education  Only child  Village - town  House - block of flats  
         0  college/bachelor degree          no         village          block of flats  
         1  college/bachelor degree          no            city          block of flats  
         2         secondary school          no            city          block of flats  
         3  college/bachelor degree         yes            city          house/bungalow  
         
         [4 rows x 150 columns]
\end{Verbatim}
    
\pagebreak
\section{Анкета}
\subsection{Музыкальные предпочтения}
\begin{enumerate}
\item Мне нравится слушать музыку: Категорически не согласен 1-2-3-4-5 Полностью согласен (целое)
\item Я предпочитаю: Медленную музыку 1-2-3-4-5 Быструю музыку (целое)
\item Dance, Disco, Funk: Не нравится вообще1-2-3-4-5 Нравится очень (целое)
\item Народная музыка: Не нравится вообще 1-2-3-4-5 люблю (целое)
\item Кантри: Не нравится вообще 1-2-3-4-5 Нравится очень (целое)
\item Классика: Не нравится вообще 1-2-3-4-5 Нравится очень (целое)
\item Мюзиклы: Не нравится вообще 1-2-3-4-5 Нравится очень (целое)
\item Поп: Не нравится вообще 1-2-3-4-5 Нравится очень (целое)
\item Рок: Не нравится вообще 1-2-3-4-5 Нравится очень (целое)
\item Металл, хард-рок: Не нравится вообще 1-2-3-4-5 Нравится очень (целое)
\item Панк: Не нравится вообще  1-2-3-4-5 Нравится очень (целое)
\item Хип-хоп, Рэп: Не нравится вообще 1-2-3-4-5 Нравится очень (целое)
\item Reggae, Ska: Не нравится вообще 1-2-3-4-5 Нравится очень (целое)
\item Swing, Jazz: не нравится вообще 1-2-3-4-5 Нравится очень (целое)
\item Рок-н-ролл: не нравится вообще 1-2-3-4-5 Нравится очень (целое)
\item Альтернативная музыка: не нравится вообще 1-2-3-4-5 Нравится очень (целое)
\item Латино: не нравится вообще 1-2-3-4-5 Нравится очень (целое)
\item Техно, Транс: не нравится вообще  1-2-3-4-5 Нравится очень (целое)
\item Опера: не нравится вообще  1-2-3-4-5 Нравится очень (целое)
\end{enumerate}
\subsection{Предпочтения в фильмах}
\begin{enumerate}
\item Мне очень нравится смотреть фильмы: Сильно не согласен 1-2-3-4-5 Полностью согласен (целое)
\item Фильмы ужасов: Не нравится вообще 1-2-3-4-5 Нравится очень (целое)
\item Триллеры: Не нравится вообще 1-2-3-4-5 Нравится очень (целое)
\item Комедии: не нравится вообще 1-2-3-4-5 Нравится очень (целое)
\item Романтические фильмы: Не нравится вообще 1-2-3-4-5 Нравится очень (целое)
\item Научно-фантастические фильмы: Не нравится вообще 1-2-3-4-5 Нравится очень (целое)
\item Военные фильмы: Не нравится вообще 1-2-3-4-5 Нравится очень (целое)
\item Сказки: Не нравится вообще 1-2-3-4-5 Нравится очень (целое)
\item Мультфильмы: Не нравится вообще 1-2-3-4-5 Наслаждайтесь очень (целое)
\item Документальные фильмы: Не нравится вообще 1-2-3-4-5 Нравится очень (целое)
\item Западные фильмы: Не нравится вообще 1-2-3-4-5 Нравится очень (целое)
\item Экшн фильмы: Не нравится вообще 1-2-3-4-5 Нравится очень (целое)
\end{enumerate}
\subsection{Интересы и хобби}
\begin{enumerate}
\item История: Не интересует 1-2-3-4-5 Очень интересно (целое)
\item Психология: не интересует 1-2-3-4-5 Очень интересно (целое)
\item Политика: Не интересует 1-2-3-4-5 Очень интересно (целое)
\item Математика: не интересует 1-2-3-4-5 Очень интересно (целое)
\item Физика: не интересует 1-2-3-4-5 Очень интересно (целое)
\item Интернет: не интересует 1-2-3-4-5 Очень интересно (целое)
\item Программное обеспечение для ПК, Оборудование: Не интересует 1-2-3-4-5 Очень интересно (целое)
\item Экономика, Менеджмент: Не интересует 1-2-3-4-5 Очень интересно (целое)
\item Биология: не интересует 1-2-3-4-5 Очень интересно (целое)
\item Химия: не интересует 1-2-3-4-5 Очень интересно (целое)
\item Чтение стихов: Не интересует 1-2-3-4-5 Очень интересно (целое)
\item География: Не интересует 1-2-3-4-5 Очень интересно (целое)
\item Иностранные языки: не интересуются 1-2-3-4-5 Очень интересно (целое)
\item Медицина: не интересует 1-2-3-4-5 Очень интересно (целое)
\item Закон: не интересует 1-2-3-4-5 Очень интересно (целое)
\item Автомобили: Не интересует 1-2-3-4-5 Очень интересно (целое)
\item Искусство: Не интересует 1-2-3-4-5 Очень интересно (целое)
\item Религия: Не интересует 1-2-3-4-5 Очень интересно (целое)
\item Мероприятия на свежем воздухе: не интересует 1-2-3-4-5 Очень интересно (целое)
\item Танцы: не интересуются 1-2-3-4-5 Очень интересно (целое)
\item Игра на музыкальных инструментах: не интересует 1-2-3-4-5 Очень интересно (целое)
\item Поэзия: Не интересно 1-2-3-4-5 Очень интересно (целое)
\item Занятия спортом: Не интересуюсь 1-2-3-4-5 Очень интересно (целое)
\item Спорт на конкурентном уровне: не интересует 1-2-3-4-5 Очень интересно (целое)
\item Садоводство: Не интересует 1-2-3-4-5 Очень интересно (целое)
\item Знаменитый образ жизни: не интересует 1-2-3-4-5 Очень интересно (целое)
\item Покупки: Не интересует 1-2-3-4-5 Очень интересно (целое)
\item Наука и техника: не интересует 1-2-3-4-5 Очень интересно (целое)
\item Театр: не интересует 1-2-3-4-5 Очень интересно (целое)
\item Общение: Не интересует 1-2-3-4-5 Очень интересно (целое)
\item Адреналин спорт: не интересует 1-2-3-4-5 Очень интересно (целое)
\item Домашние животные: Не интересуются 1-2-3-4-5 Очень интересно (целое)
\end{enumerate}
\subsection{Фобии}
\begin{enumerate}
\item Летающий: совсем не боится 1-2-3-4-5 Очень боится (целое)
\item Гром, молния: совсем не боюсь 1-2-3-4-5 Очень боится (целое)
\item Тьма: не боится совсем 1-2-3-4-5 Очень боится (целое)
\item Высота: совсем не боится 1-2-3-4-5 Очень боится (целое)
\item Пауки: совсем не боюсь 1-2-3-4-5 Очень боится (целое)
\item Змеи: совсем не боюсь 1-2-3-4-5 Очень боится (целое)
\item Крысы, мыши: совсем не боятся 1-2-3-4-5 Очень боится (целое)
\end{enumerate}
\subsection{Отношение к здоровью}
\begin{enumerate}
\item Отношение к курению: Никогда не курил - Пробовал курить - Бывший курильщик - Текущий курильщик (Номинальная)
\item Пью: Никогда - Пью в компаниях - Пью много (Номинальная)
\item Я живу очень здоровым образом жизни: сильно не согласен 1-2-3-4-5 Полностью согласен (целое)
\end{enumerate}
\subsection{Черты характера, взгляды на жизнь}
\begin{enumerate}
\item Я обращаю внимание на то, что происходит вокруг: сильно не согласен 1-2-3-4-5. Полностью согласен (целое)
\item Я стараюсь выполнять задания как можно скорее и не оставлять их до последней минуты .: Сильно не согласен 1-2-3-4-5 Полностью согласен (целое)
\item Я всегда составляю список, поэтому ничего не забываю: сильно не согласен 1-2-3-4-5 Полностью согласен (целое)
\item Я часто учусь или работаю даже в свободное время .: Сильно не согласен 1-2-3-4-5 Полностью согласен (целое)
\item Я смотрю на вещи с разных точек зрения, прежде чем идти вперед. Полностью не согласен 1-2-3-4-5. Полностью согласен (целое)
\item Я считаю, что плохие люди пострадают в один прекрасный день, и хорошие люди будут вознаграждены. Полностью не согласен 1-2-3-4-5. Полностью согласен (целое)
\item Я надёжен на работе и всегда выполняю все заданные мне задачи: Полностью не согласен 1-2-3-4-5 Полностью согласен (целое)
\item Я всегда соблюдаю свои обещания: сильно не согласен 1-2-3-4-5 Полностью согласен (целое)
\item Я могу быстро попасть на кого-то, а затем полностью потерять интерес. Полностью не согласен 1-2-3-4-5. Полностью согласен (целое)
\item Я бы предпочел иметь много друзей, чем много денег. Полностью не согласен 1-2-3-4-5. Полностью согласен (целое)
\item Я всегда стараюсь быть самым смешным: сильно не согласен 1-2-3-4-5 Полностью согласен (целое)
\item Иногда я могу столкнуться с двумя лицами: Полностью не согласен 1-2-3-4-5 Полностью согласен (целое)
\item Я повредил вещи в прошлом, когда сердился .: Сильно не согласен 1-2-3-4-5 Полностью согласен (целое)
\item Я не тороплюсь, чтобы принимать решения: сильно не согласен 1-2-3-4-5 Полностью согласен (целое)
\item Я всегда стараюсь голосовать на выборах: сильно не согласен 1-2-3-4-5 Полностью согласен (целое)
\item Я часто думаю и сожалею о принимаемых мной решениях: Полностью не согласен 1-2-3-4-5 Полностью согласен (целое)
\item Я могу сказать, слушают ли меня люди или нет, когда я говорю с ними: Полностью не согласен 1-2-3-4-5 Полностью согласен (целое)
\item Я ипохондрик. Полностью не согласен 1-2-3-4-5. Полностью согласен (целое)
\item Я - эмфатичный человек. Сильно не согласен 1-2-3-4-5 Полностью согласен (целое)
\item Я ем, потому что должен. Я не наслаждаюсь едой и еду так быстро, как могу: Категорически не согласен 1-2-3-4-5 Полностью согласен (целое)
\item Я стараюсь отдать столько, сколько я могу, другим людям на Рождество: сильно не согласен 1-2-3-4-5 Полностью согласен (целое)
\item Мне не нравится встречаться с животными: Полностью не согласен 1-2-3-4-5 Полностью согласен (целое)
\item Я ухаживаю за вещами, которые я заимствовал у других: сильно не согласен 1-2-3-4-5 Полностью согласен (целое)
\item Я чувствую себя одиноким в жизни: Сильно не согласен 1-2-3-4-5 Полностью согласен (целое)
\item Раньше я учился в школе: Сильно не согласен 1-2-3-4-5 Полностью согласен (целое)
\item Я беспокоюсь о своем здоровье: сильно не согласен 1-2-3-4-5 Полностью согласен (целое)
\item Хотел бы я изменить прошлое из-за того, что я сделал: Полностью не согласен 1-2-3-4-5 Сильно соглашаюсь (целое)
\item Я верю в Бога: Полностью не согласен 1-2-3-4-5 Полностью согласен (целое)
\item У меня всегда хорошие мечты: сильно не согласен 1-2-3-4-5 Полностью согласен (целое)
\item Я всегда отдаю благотворительность. Полностью не согласен 1-2-3-4-5. Полностью согласен (целое)
\item У меня много друзей: сильно не согласен 1-2-3-4-5 Полностью согласен (целое)
\item Сроки: Я часто бываю на ранней стадии. - Я всегда вовремя. - Я часто опаздываю. (Категорично)
\item Вы лжете другим: Никогда. - Только чтобы не причинять кому-либо вреда. Иногда. - Каждый раз мне это подходит. (Номинальная)
\item Я очень терпелив: сильно не согласен 1-2-3-4-5 Полностью согласен (целое)
\item Я могу быстро адаптироваться к новой среде .: Сильно не согласен 1-2-3-4-5 Полностью согласен (целое)
\item Мое настроение меняется быстро: сильно не согласен 1-2-3-4-5 Полностью согласен (целое)
\item Я хорошо воспитан, и я ухаживаю за внешностью: Полностью не согласен 1-2-3-4-5 Полностью согласен (целое)
\item Мне нравится встречаться с новыми людьми .: Сильно не согласен 1-2-3-4-5 Полностью согласен (целое)
\item Я всегда позволяю другим людям узнать о моих достижениях: Полностью не согласен 1-2-3-4-5 Полностью согласен (целое)
\item Я думаю, тщательно, прежде чем отвечать на любые важные письма .: Полностью не согласен 1-2-3-4-5 Полностью согласен (целое)
\item Я люблю детей: сильно не согласен 1-2-3-4-5 Полностью согласен (целое)
\item Я не боюсь высказать свое мнение, если я сильно против чего либо: Полностью не согласен 1-2-3-4-5 Полностью согласен (целое)
\item Я могу рассердиться очень легко: сильно не согласен 1-2-3-4-5 Полностью согласен (целое)
\item Я всегда убеждаюсь, что я общаюсь с нужными людьми: сильно не согласен 1-2-3-4-5 Полностью согласен (целое)
\item Я должен быть хорошо подготовлен перед публичным выступлением: Полностью не согласен 1-2-3-4-5 Полностью согласен (целое)
\item Я виноват в том, люди меня не любят. Полностью не согласен 1-2-3-4-5. Полностью согласен (целое)
\item Я плачу, когда чувствую себя, или все идет не так. Абсолютно не согласен 1-2-3-4-5. Полностью согласен (целое)
\item Я на 100\% доволен своей жизнью. Полностью не согласен 1-2-3-4-5. Полностью согласен (целое)
\item Я всегда полна жизни и энергии: Полностью не согласен 1-2-3-4-5 Полностью согласен (целое)
\item Я предпочитаю крупных опасных собак более мелким, более спокойным собакам: Полностью не согласен 1-2-3-4-5 Полностью согласен (целое)
\item Я считаю, что все мои черты характера положительные: сильно не согласны 1-2-3-4-5. Полностью согласен (целое)
\item Если я найду что-то, что не принадлежит мне, я передам его. Полностью не согласен 1-2-3-4-5. Полностью согласен (целое)
\item Мне очень трудно вставать утром: сильно не согласен 1-2-3-4-5 Полностью согласен (целое)
\item У меня много разных увлечений и интересов: Полностью не согласен 1-2-3-4-5 Полностью согласен (целое)
\item Я всегда слушаю совет моих родителей: Полностью не согласен 1-2-3-4-5. Полностью согласен (целое)
\item Мне нравится участвовать в опросах: сильно не согласен 1-2-3-4-5 Полностью согласен (целое)
\item Сколько времени вы проводите в Интернете ?: Не провожу - Менее часа в день - Несколько часов в день - Большая часть дня (Номинальная)
\end{enumerate}
\subsection{Отношение к деньгам}
\begin{enumerate}
\item Я могу сэкономить все, что смогу. Полностью не согласен 1-2-3-4-5. Полностью согласен (целое)
\item Мне нравится ходить в крупные торговые центры .: Сильно не согласен 1-2-3-4-5 Полностью согласен (целое)
\item Я предпочитаю фирменную одежду: сильно не согласен 1-2-3-4-5 Полностью согласен (целое)
\item Я трачу много денег на вечеринки и общение: Сильно не согласен 1-2-3-4-5 Полностью согласен (целое)
\item Я трачу много денег на внешний вид: сильно не согласен 1-2-3-4-5 Полностью согласен (целое)
\item Я трачу много денег на гаджеты: Полностью не согласен 1-2-3-4-5 Полностью согласен (целое)
\item Я буду с удовольствием  платить больше денег за хорошее, качественное или здоровое питание: Полностью не согласен 1-2-3-4-5 Полностью согласен (целое)
\end{enumerate}
\subsection{Данные о себе}
\begin{enumerate}
\item Возраст: (целое)
\item Высота: (целое)
\item Вес: (целое)
\item Сколько у вас братьев и сестер ?: (целое)
\item Пол: Женский - Мужской (Номинальная)
\item Я: Левая рука - Правая (Номинальная)
\item Высшее образование: В настоящее время ученик начальной школы - Начальная школа - Средняя школа - Колледж / Степень бакалавра (Номинальная)
\item Я единственный ребенок: Нет - Да (Номинальная)
\item Я провел большую часть своего детства в: Городе -  деревне (Номинальная)
\item Я прожил большую часть своего детства в: доме  - многоквартирном доме (Номинальная)
\end{enumerate}
\pagebreak

\section{Описательные статистики}
\subsection{Основные численные характеристики}
Из-за большого количества признаков, рассмотрим лишь подможество из множества ответов, а именно рассмотрим
предпочтения в музыке. Для каждой характериски расчитаем 

\begin{itemize}
\item mean – средняя величина из исходных значений $\bar{x} = \frac{x_1+x_2+\cdots +x_n}{n}$;
\item std  –  стандартное отклонение, мера того, насколько широко разбросаны точки данных относительно их средних $s = \sqrt{\frac{1}{N-1} \sum_{i=1}^N (x_i - \overline{x})^2}$
\item min, max -- минимальное и максимальное значение
\item 25\% , 50\% , 75\% квантили соответствующих уровней. Значение, которое заданная случайная величина не превышает с фиксированной вероятностью. Верхняя квантиль включает 25\% наибольших чисел в наборе, нижняя, соответственно, 25\% наименьших чисел в наборе, 50\% соответсвует медиане выборки
\item skewness – коэффициент асимметрии, который показывает, насколько симметрично распределение данной случайной величины. Если коэффициент асимметрии положителен, то отклонение происходит в сторону положительных значений, в ином случае – отрицательных
\item kurtosis – эксцесс – мера крутости кривой распределения. Кривая распределения может быть островершинной, плосковершинной, средне вершинной. Эти четыре момента составляют набор особенностей распределения при анализе данных. Для нормального распределения А=0, Е=0. Положительный эксцесс обозначает относительно остроконечное распределение. Отрицательный эксцесс обозначает относительно сглаженное распределение
\item mode – мода, наиболее часто встречаемое значение
\item NAs -- количество пропущенных данных. 
\end{itemize}
  
    \begin{Verbatim}[commandchars=\\\{\}]
{\color{incolor}In [{\color{incolor}11}]:} \PY{k+kn}{import} \PY{n+nn}{pandas} \PY{k}{as} \PY{n+nn}{pd}
         \PY{k+kn}{import} \PY{n+nn}{numpy} \PY{k}{as} \PY{n+nn}{np}
\end{Verbatim}

%описательные
\begin{Verbatim}[commandchars=\\\{\}]
{\color{incolor}In [{\color{incolor}4}]:} \PY{k+kn}{from} \PY{n+nn}{pandas} \PY{k}{import} \PY{n}{Series}
        \PY{k}{def} \PY{n+nf}{customDescribe}\PY{p}{(}\PY{n}{x}\PY{p}{)}\PY{p}{:}
            \PY{n}{data} \PY{o}{=} \PY{p}{[}\PY{n}{x}\PY{o}{.}\PY{n}{mean}\PY{p}{(}\PY{p}{)}\PY{p}{,} \PY{n}{x}\PY{o}{.}\PY{n}{std}\PY{p}{(}\PY{p}{)}\PY{p}{,} \PY{n}{x}\PY{o}{.}\PY{n}{min}\PY{p}{(}\PY{p}{)}\PY{p}{,} \PY{n}{x}\PY{o}{.}\PY{n}{quantile}\PY{p}{(}\PY{l+m+mf}{0.25}\PY{p}{)}\PY{p}{,} \PY{n}{x}\PY{o}{.}\PY{n}{median}\PY{p}{(}\PY{p}{)}\PY{p}{,} 
                    \PY{n}{x}\PY{o}{.}\PY{n}{quantile}\PY{p}{(}\PY{l+m+mf}{0.75}\PY{p}{)}\PY{p}{,} \PY{n}{x}\PY{o}{.}\PY{n}{max}\PY{p}{(}\PY{p}{)}\PY{p}{,} \PY{n}{x}\PY{o}{.}\PY{n}{skew}\PY{p}{(}\PY{p}{)}\PY{p}{,} \PY{n}{x}\PY{o}{.}\PY{n}{kurtosis}\PY{p}{(}\PY{p}{)}\PY{p}{,} \PY{n}{x}\PY{o}{.}\PY{n}{mode}\PY{p}{(}\PY{p}{)}\PY{o}{.}\PY{n}{max}\PY{p}{(}\PY{p}{)}\PY{p}{,} 
                    \PY{n}{x}\PY{o}{.}\PY{n}{isnull}\PY{p}{(}\PY{p}{)}\PY{o}{.}\PY{n}{sum}\PY{p}{(}\PY{p}{)}\PY{p}{]}
            \PY{n}{names} \PY{o}{=} \PY{p}{[}\PY{l+s+s1}{\PYZsq{}}\PY{l+s+s1}{mean}\PY{l+s+s1}{\PYZsq{}}\PY{p}{,} \PY{l+s+s1}{\PYZsq{}}\PY{l+s+s1}{std}\PY{l+s+s1}{\PYZsq{}}\PY{p}{,} \PY{l+s+s1}{\PYZsq{}}\PY{l+s+s1}{min}\PY{l+s+s1}{\PYZsq{}}\PY{p}{,} \PY{l+s+s1}{\PYZsq{}}\PY{l+s+s1}{25}\PY{l+s+s1}{\PYZpc{}}\PY{l+s+s1}{\PYZsq{}}\PY{p}{,} \PY{l+s+s1}{\PYZsq{}}\PY{l+s+s1}{50}\PY{l+s+s1}{\PYZpc{}}\PY{l+s+s1}{\PYZsq{}}\PY{p}{,} \PY{l+s+s1}{\PYZsq{}}\PY{l+s+s1}{75}\PY{l+s+s1}{\PYZpc{}}\PY{l+s+s1}{\PYZsq{}}\PY{p}{,} \PY{l+s+s1}{\PYZsq{}}\PY{l+s+s1}{max}\PY{l+s+s1}{\PYZsq{}}\PY{p}{,}
             \PY{l+s+s1}{\PYZsq{}}\PY{l+s+s1}{skewness}\PY{l+s+s1}{\PYZsq{}}\PY{p}{,} \PY{l+s+s1}{\PYZsq{}}\PY{l+s+s1}{kurtosis}\PY{l+s+s1}{\PYZsq{}}\PY{p}{,} \PY{l+s+s1}{\PYZsq{}}\PY{l+s+s1}{mode}\PY{l+s+s1}{\PYZsq{}}\PY{p}{,} \PY{l+s+s1}{\PYZsq{}}\PY{l+s+s1}{NAs}\PY{l+s+s1}{\PYZsq{}}\PY{p}{]}
            \PY{k}{return} \PY{n}{Series}\PY{p}{(}\PY{n}{data}\PY{p}{,} \PY{n}{index}\PY{o}{=}\PY{n}{names}\PY{p}{)}
        
        \PY{n}{names} \PY{o}{=} \PY{n}{pd}\PY{o}{.}\PY{n}{read\PYZus{}csv}\PY{p}{(}\PY{l+s+s1}{\PYZsq{}}\PY{l+s+s1}{columns.csv}\PY{l+s+s1}{\PYZsq{}}\PY{p}{)}
        \PY{n}{df} \PY{o}{=} \PY{n}{pd}\PY{o}{.}\PY{n}{read\PYZus{}csv}\PY{p}{(}\PY{l+s+s1}{\PYZsq{}}\PY{l+s+s1}{responses.csv}\PY{l+s+s1}{\PYZsq{}}\PY{p}{)}
        \PY{n}{music} \PY{o}{=} \PY{n}{df}\PY{o}{.}\PY{n}{iloc}\PY{p}{[}\PY{p}{:}\PY{p}{,}\PY{l+m+mi}{2}\PY{p}{:}\PY{l+m+mi}{18}\PY{p}{]}
        \PY{n}{music}\PY{o}{.}\PY{n}{apply}\PY{p}{(}\PY{n}{customDescribe}\PY{p}{)}
\end{Verbatim}
\begin{Verbatim}[commandchars=\\\{\}]
{\color{outcolor}Out[{\color{outcolor}4}]:}              Dance      Folk   Country  Classical music   Musical       Pop  \textbackslash{}
        mean      3.113320  2.288557  2.123383         2.956132  2.761905  3.471698   
        std       1.170568  1.138916  1.076136         1.252570  1.260845  1.161400   
        min       1.000000  1.000000  1.000000         1.000000  1.000000  1.000000   
        25\%       2.000000  1.000000  1.000000         2.000000  2.000000  3.000000   
        50\%       3.000000  2.000000  2.000000         3.000000  3.000000  4.000000   
        75\%       4.000000  3.000000  3.000000         4.000000  4.000000  4.000000   
        max       5.000000  5.000000  5.000000         5.000000  5.000000  5.000000   
        skewness -0.045760  0.694783  0.795798         0.107357  0.219951 -0.383317   
        kurtosis -0.803331 -0.216416 -0.037576        -0.969287 -0.928080 -0.704309   
        mode      3.000000  2.000000  2.000000         3.000000  3.000000  4.000000   
        NAs       4.000000  5.000000  5.000000         7.000000  2.000000  3.000000   
        
                      Rock  Metal or Hardrock      Punk  Hiphop, Rap  Reggae, Ska  \textbackslash{}
        mean      3.761952           2.361470  2.456088     2.910537     2.769691   
        std       1.184861           1.372995  1.301105     1.375677     1.214434   
        min       1.000000           1.000000  1.000000     1.000000     1.000000   
        25\%       3.000000           1.000000  1.000000     2.000000     2.000000   
        50\%       4.000000           2.000000  2.000000     3.000000     3.000000   
        75\%       5.000000           3.000000  3.000000     4.000000     4.000000   
        max       5.000000           5.000000  5.000000     5.000000     5.000000   
        skewness -0.702586           0.604915  0.441427     0.037217     0.156497   
        kurtosis -0.419187          -0.934732 -0.959379    -1.250059    -0.900509   
        mode      5.000000           1.000000  1.000000     4.000000     3.000000   
        NAs       6.000000           3.000000  8.000000     4.000000     7.000000   
        
                  Swing, Jazz  Rock n roll  Alternative    Latino  Techno, Trance  
        mean         2.759960     3.141575     2.828514  2.842315        2.338983  
        std          1.257936     1.237269     1.347173  1.327902        1.324099  
        min          1.000000     1.000000     1.000000  1.000000        1.000000  
        25\%          2.000000     2.000000     2.000000  2.000000        1.000000  
        50\%          3.000000     3.000000     3.000000  3.000000        2.000000  
        75\%          4.000000     4.000000     4.000000  4.000000        3.000000  
        max          5.000000     5.000000     5.000000  5.000000        5.000000  
        skewness     0.146457    -0.108936     0.162211  0.188489        0.569644  
        kurtosis    -0.997739    -0.917436    -1.129404 -1.099347       -0.906037  
        mode         3.000000     3.000000     3.000000  2.000000        1.000000  
        NAs          6.000000     7.000000     7.000000  8.000000        7.000000  
\end{Verbatim}
 

\subsection{Визуализация данных}

%графички
\begin{Verbatim}[commandchars=\\\{\}]
{\color{incolor}In [{\color{incolor}5}]:} \PY{n}{music} \PY{o}{=} \PY{n}{music}\PY{o}{.}\PY{n}{dropna}\PY{p}{(}\PY{p}{)}
\end{Verbatim}


    \begin{Verbatim}[commandchars=\\\{\}]
{\color{incolor}In [{\color{incolor}9}]:} \PY{k+kn}{import} \PY{n+nn}{seaborn} \PY{k}{as} \PY{n+nn}{sns}
        \PY{k+kn}{import} \PY{n+nn}{matplotlib}\PY{n+nn}{.}\PY{n+nn}{pyplot} \PY{k}{as} \PY{n+nn}{plt}
        \PY{n}{plt\PYZus{}dict} \PY{o}{=} \PY{p}{\PYZob{}}\PY{p}{\PYZcb{}}
        
        \PY{k}{for} \PY{n}{i} \PY{o+ow}{in} \PY{n+nb}{range}\PY{p}{(}\PY{l+m+mi}{0}\PY{p}{,}\PY{n+nb}{len}\PY{p}{(}\PY{n}{music}\PY{o}{.}\PY{n}{columns}\PY{p}{)}\PY{p}{)}\PY{p}{:}
            \PY{n}{plt\PYZus{}dict}\PY{o}{.}\PY{n}{update}\PY{p}{(}\PY{p}{\PYZob{}}\PY{n}{i}\PY{p}{:}\PY{n}{music}\PY{o}{.}\PY{n}{columns}\PY{p}{[}\PY{n}{i}\PY{p}{]}\PY{p}{\PYZcb{}}\PY{p}{)}
        
        \PY{n}{fig}\PY{p}{,} \PY{n}{ax} \PY{o}{=} \PY{n}{plt}\PY{o}{.}\PY{n}{subplots}\PY{p}{(}\PY{l+m+mi}{4}\PY{p}{,}\PY{l+m+mi}{4}\PY{p}{,}\PY{n}{figsize}\PY{o}{=}\PY{p}{(}\PY{l+m+mi}{15}\PY{p}{,}\PY{l+m+mi}{15}\PY{p}{)}\PY{p}{,} \PY{n}{sharey}\PY{o}{=}\PY{k+kc}{True}\PY{p}{,} \PY{n}{sharex}\PY{o}{=}\PY{k+kc}{True}\PY{p}{)}
        \PY{n}{x} \PY{o}{=} \PY{p}{[}\PY{l+m+mi}{1}\PY{p}{,}\PY{l+m+mi}{2}\PY{p}{,}\PY{l+m+mi}{3}\PY{p}{,}\PY{l+m+mi}{4}\PY{p}{,}\PY{l+m+mi}{5}\PY{p}{]}
        \PY{n}{initial} \PY{o}{=} \PY{l+m+mi}{0}
        
        \PY{k}{for} \PY{n}{i} \PY{o+ow}{in} \PY{n+nb}{range}\PY{p}{(}\PY{l+m+mi}{4}\PY{p}{)}\PY{p}{:}
            \PY{k}{for} \PY{n}{j} \PY{o+ow}{in} \PY{n+nb}{range}\PY{p}{(}\PY{l+m+mi}{4}\PY{p}{)}\PY{p}{:}
                \PY{n}{y} \PY{o}{=} \PY{n}{music}\PY{p}{[}\PY{n}{plt\PYZus{}dict}\PY{p}{[}\PY{n}{initial}\PY{p}{]}\PY{p}{]}\PY{o}{.}\PY{n}{value\PYZus{}counts}\PY{p}{(}\PY{p}{)}\PY{o}{.}\PY{n}{to\PYZus{}dict}\PY{p}{(}\PY{p}{)}
                \PY{n}{ax}\PY{p}{[}\PY{n}{i}\PY{p}{,}\PY{n}{j}\PY{p}{]}\PY{o}{.}\PY{n}{bar}\PY{p}{(}\PY{n}{y}\PY{o}{.}\PY{n}{keys}\PY{p}{(}\PY{p}{)}\PY{p}{,} \PY{n}{y}\PY{o}{.}\PY{n}{values}\PY{p}{(}\PY{p}{)}\PY{p}{)}
                \PY{n}{ax}\PY{p}{[}\PY{n}{i}\PY{p}{,}\PY{n}{j}\PY{p}{]}\PY{o}{.}\PY{n}{set\PYZus{}ylabel}\PY{p}{(}\PY{l+s+s1}{\PYZsq{}}\PY{l+s+s1}{\PYZsq{}}\PY{p}{)}
                \PY{n}{ax}\PY{p}{[}\PY{n}{i}\PY{p}{,}\PY{n}{j}\PY{p}{]}\PY{o}{.}\PY{n}{set\PYZus{}xlabel}\PY{p}{(}\PY{l+s+s1}{\PYZsq{}}\PY{l+s+s1}{\PYZsq{}}\PY{p}{)}
                \PY{n}{ax}\PY{p}{[}\PY{n}{i}\PY{p}{,}\PY{n}{j}\PY{p}{]}\PY{o}{.}\PY{n}{set\PYZus{}xticklabels}\PY{p}{(}\PY{n}{labels}\PY{o}{=}\PY{n}{np}\PY{o}{.}\PY{n}{arange}\PY{p}{(}\PY{l+m+mi}{0}\PY{p}{,}\PY{l+m+mi}{6}\PY{p}{)}\PY{p}{,} \PY{n}{fontsize}\PY{o}{=}\PY{l+m+mi}{10}\PY{p}{)}
                \PY{n}{ax}\PY{p}{[}\PY{n}{i}\PY{p}{,}\PY{n}{j}\PY{p}{]}\PY{o}{.}\PY{n}{set\PYZus{}yticklabels}\PY{p}{(}\PY{n}{labels}\PY{o}{=}\PY{n}{np}\PY{o}{.}\PY{n}{arange}\PY{p}{(}\PY{l+m+mi}{0}\PY{p}{,}\PY{l+m+mi}{601}\PY{p}{,}\PY{l+m+mi}{100}\PY{p}{)}\PY{p}{,} \PY{n}{fontsize}\PY{o}{=}\PY{l+m+mi}{10}\PY{p}{)}
                \PY{n}{ax}\PY{p}{[}\PY{n}{i}\PY{p}{,}\PY{n}{j}\PY{p}{]}\PY{o}{.}\PY{n}{set\PYZus{}title}\PY{p}{(}\PY{n}{plt\PYZus{}dict}\PY{p}{[}\PY{n}{initial}\PY{p}{]}\PY{p}{,} \PY{n}{fontsize}\PY{o}{=}\PY{l+m+mi}{10}\PY{p}{)}
                \PY{n}{ax}\PY{p}{[}\PY{n}{i}\PY{p}{,}\PY{n}{j}\PY{p}{]}\PY{o}{.}\PY{n}{set\PYZus{}xlim}\PY{p}{(}\PY{o}{.}\PY{l+m+mi}{5}\PY{p}{,}\PY{l+m+mi}{6}\PY{p}{)}
                \PY{n}{ax}\PY{p}{[}\PY{n}{i}\PY{p}{,}\PY{n}{j}\PY{p}{]}\PY{o}{.}\PY{n}{set\PYZus{}ylim}\PY{p}{(}\PY{l+m+mi}{0}\PY{p}{,}\PY{l+m+mi}{600}\PY{p}{)}
                \PY{n}{initial} \PY{o}{+}\PY{o}{=} \PY{l+m+mi}{1}
        \PY{n}{plt}\PY{o}{.}\PY{n}{show}\PY{p}{(}\PY{p}{)}
\end{Verbatim}


    \begin{center}
    \adjustimage{max size={0.9\linewidth}{0.9\paperheight}}{output_3_0.png}
    \end{center}
    { \hspace*{\fill} \\}
    
\subsection{Вывод}
Как видно, кривые распределений плосковерхие, при этом для большинсва графиков четко прослеживается 
тяготение к одному из крайних допустимых значений. Это связано, во-первых, с тем, что данные носят дискретный
характер, а во-вторых, сам характер данных (предпочтение в музыке) носят всеобщий и стихийный характер. Стиль или более-менее устраивает всех -- альтернативная музыка, теряет популярность, как, например фолк, имеет преданных фанатав, как рок,
а может вызывать неприязнь, как техно. 

\section{Проверка нормальности распределения}
Построим normal probability plot для всех характеристик наших данных (ожидаемое и представленное 
значение). Как видно, распределены не нормально. Это же следует из предоставленного окружением 
scipy теста normaltest, который, в частности возвращает p-value, которое показывает вероятность 
отклонения нулевой гипотезы, которая состоит в том, что данные распределены равномерно. 


\begin{Verbatim}[commandchars=\\\{\}]
{\color{incolor}In [{\color{incolor}38}]:} \PY{k+kn}{from} \PY{n+nn}{scipy}\PY{n+nn}{.}\PY{n+nn}{stats}\PY{n+nn}{.}\PY{n+nn}{mstats} \PY{k}{import} \PY{n}{normaltest}
         \PY{k+kn}{from} \PY{n+nn}{scipy}\PY{n+nn}{.}\PY{n+nn}{stats} \PY{k}{import} \PY{n}{probplot}
         \PY{k+kn}{import} \PY{n+nn}{seaborn} \PY{k}{as} \PY{n+nn}{sns}
         \PY{k+kn}{import} \PY{n+nn}{matplotlib}\PY{n+nn}{.}\PY{n+nn}{pyplot} \PY{k}{as} \PY{n+nn}{plt}
         \PY{n}{plt\PYZus{}dict} \PY{o}{=} \PY{p}{\PYZob{}}\PY{p}{\PYZcb{}}
         
         \PY{k}{for} \PY{n}{i} \PY{o+ow}{in} \PY{n+nb}{range}\PY{p}{(}\PY{l+m+mi}{0}\PY{p}{,}\PY{n+nb}{len}\PY{p}{(}\PY{n}{music}\PY{o}{.}\PY{n}{columns}\PY{p}{)}\PY{p}{)}\PY{p}{:}
             \PY{n}{plt\PYZus{}dict}\PY{o}{.}\PY{n}{update}\PY{p}{(}\PY{p}{\PYZob{}}\PY{n}{i}\PY{p}{:}\PY{n}{music}\PY{o}{.}\PY{n}{columns}\PY{p}{[}\PY{n}{i}\PY{p}{]}\PY{p}{\PYZcb{}}\PY{p}{)}
         
         \PY{n}{fig}\PY{p}{,} \PY{n}{ax} \PY{o}{=} \PY{n}{plt}\PY{o}{.}\PY{n}{subplots}\PY{p}{(}\PY{l+m+mi}{4}\PY{p}{,}\PY{l+m+mi}{4}\PY{p}{,}\PY{n}{figsize}\PY{o}{=}\PY{p}{(}\PY{l+m+mi}{15}\PY{p}{,}\PY{l+m+mi}{15}\PY{p}{)}\PY{p}{,} \PY{n}{sharey}\PY{o}{=}\PY{k+kc}{True}\PY{p}{,} \PY{n}{sharex}\PY{o}{=}\PY{k+kc}{True}\PY{p}{)}
         \PY{n}{x} \PY{o}{=} \PY{p}{[}\PY{l+m+mi}{1}\PY{p}{,}\PY{l+m+mi}{2}\PY{p}{,}\PY{l+m+mi}{3}\PY{p}{,}\PY{l+m+mi}{4}\PY{p}{,}\PY{l+m+mi}{5}\PY{p}{]}
         \PY{n}{initial} \PY{o}{=} \PY{l+m+mi}{0}
         
         \PY{k}{for} \PY{n}{i} \PY{o+ow}{in} \PY{n+nb}{range}\PY{p}{(}\PY{l+m+mi}{4}\PY{p}{)}\PY{p}{:}
             \PY{k}{for} \PY{n}{j} \PY{o+ow}{in} \PY{n+nb}{range}\PY{p}{(}\PY{l+m+mi}{4}\PY{p}{)}\PY{p}{:}
                 \PY{n}{probplot}\PY{p}{(}\PY{n}{music}\PY{p}{[}\PY{n}{plt\PYZus{}dict}\PY{p}{[}\PY{n}{initial}\PY{p}{]}\PY{p}{]}\PY{p}{,} \PY{n}{plot}\PY{o}{=}\PY{n}{ax}\PY{p}{[}\PY{n}{i}\PY{p}{,}\PY{n}{j}\PY{p}{]}\PY{p}{)}
                 \PY{n}{ax}\PY{p}{[}\PY{n}{i}\PY{p}{,}\PY{n}{j}\PY{p}{]}\PY{o}{.}\PY{n}{set\PYZus{}ylabel}\PY{p}{(}\PY{l+s+s1}{\PYZsq{}}\PY{l+s+s1}{\PYZsq{}}\PY{p}{)}
                 \PY{n}{ax}\PY{p}{[}\PY{n}{i}\PY{p}{,}\PY{n}{j}\PY{p}{]}\PY{o}{.}\PY{n}{set\PYZus{}xlabel}\PY{p}{(}\PY{l+s+s1}{\PYZsq{}}\PY{l+s+s1}{\PYZsq{}}\PY{p}{)}
                 \PY{n}{ax}\PY{p}{[}\PY{n}{i}\PY{p}{,}\PY{n}{j}\PY{p}{]}\PY{o}{.}\PY{n}{set\PYZus{}xticklabels}\PY{p}{(}\PY{n}{labels}\PY{o}{=}\PY{n}{np}\PY{o}{.}\PY{n}{arange}\PY{p}{(}\PY{o}{\PYZhy{}}\PY{l+m+mi}{20}\PY{p}{,}\PY{l+m+mi}{20}\PY{p}{)}\PY{p}{,} \PY{n}{fontsize}\PY{o}{=}\PY{l+m+mi}{10}\PY{p}{)}
                 \PY{n}{ax}\PY{p}{[}\PY{n}{i}\PY{p}{,}\PY{n}{j}\PY{p}{]}\PY{o}{.}\PY{n}{set\PYZus{}yticklabels}\PY{p}{(}\PY{n}{labels}\PY{o}{=}\PY{n}{np}\PY{o}{.}\PY{n}{arange}\PY{p}{(}\PY{l+m+mi}{0}\PY{p}{,}\PY{l+m+mi}{10}\PY{p}{,}\PY{l+m+mi}{1}\PY{p}{)}\PY{p}{,} \PY{n}{fontsize}\PY{o}{=}\PY{l+m+mi}{10}\PY{p}{)}
                 \PY{n}{ax}\PY{p}{[}\PY{n}{i}\PY{p}{,}\PY{n}{j}\PY{p}{]}\PY{o}{.}\PY{n}{set\PYZus{}title}\PY{p}{(}\PY{n}{plt\PYZus{}dict}\PY{p}{[}\PY{n}{initial}\PY{p}{]}\PY{p}{,} \PY{n}{fontsize}\PY{o}{=}\PY{l+m+mi}{10}\PY{p}{)}
                 \PY{n}{ax}\PY{p}{[}\PY{n}{i}\PY{p}{,}\PY{n}{j}\PY{p}{]}\PY{o}{.}\PY{n}{set\PYZus{}ylim}\PY{p}{(}\PY{l+m+mi}{0}\PY{p}{,}\PY{l+m+mi}{6}\PY{p}{)}
                 
                 \PY{n}{initial} \PY{o}{+}\PY{o}{=} \PY{l+m+mi}{1}
         \PY{n}{plt}\PY{o}{.}\PY{n}{show}\PY{p}{(}\PY{p}{)}
         
         \PY{n}{normaltest}\PY{p}{(}\PY{n}{music}\PY{p}{)}\PY{o}{.}\PY{n}{pvalue}
\end{Verbatim}



    \begin{center}
    \adjustimage{max size={0.9\linewidth}{0.9\paperheight}}{output_4_0.png}
    \end{center}
    { \hspace*{\fill} \\}
    
\begin{Verbatim}[commandchars=\\\{\}]
{\color{outcolor}Out[{\color{outcolor}38}]:} array([  1.73933578e-018,   9.70589103e-015,   9.08852599e-018,
                  3.03938325e-039,   1.85007144e-036,   1.81645640e-017,
                  1.15204030e-016,   6.09948269e-042,   3.13553153e-039,
                  2.84113677e-210,   7.36973614e-024,   2.01589772e-045,
                  2.74815258e-029,   1.90506599e-087,   1.42014654e-076,
                  2.82482142e-037])
\end{Verbatim}
 
\section{Факторный анализ}
\subsection{Постановка задачи} 
Попробуем уменьшить предпочтения в музыке в несколько удобных, легко интерпретируемых
 факторов, которые лучше помогут
понять ответы респондендентов. Подобный подход может быть применен по любой другой схожей
категории ответов (по фильмам, хобби и т.д). 

\subsection{Корреляции} 
Для начала посмотрим, как данные коррелируют друг с другом, воспользовавшись 
коэфициентом Пирсона $\mathbf{r}_{XY} = \frac{\mathbf{cov}_{XY}}{\mathbf{\sigma}_{X}{\sigma}_{Y}}= \frac{\sum (X-\bar{X})(Y-\bar{Y})}{\sqrt{\sum (X-\bar{X})^2}\sqrt{\sum (Y-\bar{Y})^2}}$
           
    \begin{Verbatim}[commandchars=\\\{\}]
{\color{incolor}In [{\color{incolor}15}]:} \PY{k+kn}{import} \PY{n+nn}{seaborn} \PY{k}{as} \PY{n+nn}{sbn}
         \PY{k+kn}{import} \PY{n+nn}{matplotlib}\PY{n+nn}{.}\PY{n+nn}{pyplot} \PY{k}{as} \PY{n+nn}{plt}
         \PY{n}{corr} \PY{o}{=} \PY{n}{music}\PY{o}{.}\PY{n}{corr}\PY{p}{(}\PY{p}{)} 
         \PY{n}{sbn}\PY{o}{.}\PY{n}{heatmap}\PY{p}{(}\PY{n}{corr}\PY{p}{,} 
                     \PY{n}{xticklabels}\PY{o}{=}\PY{n}{corr}\PY{o}{.}\PY{n}{columns}\PY{o}{.}\PY{n}{values}\PY{p}{,}
                     \PY{n}{yticklabels}\PY{o}{=}\PY{n}{corr}\PY{o}{.}\PY{n}{columns}\PY{o}{.}\PY{n}{values}\PY{p}{,}
                     \PY{n}{vmax}\PY{o}{=}\PY{o}{.}\PY{l+m+mi}{8}\PY{p}{,} \PY{n}{square}\PY{o}{=}\PY{k+kc}{True}\PY{p}{)}
         \PY{n}{plt}\PY{o}{.}\PY{n}{show}\PY{p}{(}\PY{p}{)}
\end{Verbatim}


    \begin{center}
    \adjustimage{max size={0.8\linewidth}{0.8\paperheight}}{output_2_0.png}
    \end{center}
    { \hspace*{\fill} \\}
    
Данные связанны между собой довольно очевидным образом, например классическая музыка и опера, рок, панк
и тяжелый метал, а так же наблюдаются другие, не менее очевидные корреляции. 

\subsection{Описание метода} 
Проведем факторый анализ. Он имеет простую линейную форму 
$$z_{ai}=\sum_p \ell_{ap}F_{pi}+\varepsilon_{ai},$$ где $F_{pi}$ это общие факторы, а $\varepsilon_{ai}$ 
характерный фактор $i$-ого показателя. Дисперсия любого исходного показателя состоит из общности и характерности. Общность характеризует ту часть дисперсии исходного показателя, которая объясняется общими факторами. 
Ставится задача по исходным показателям подобрать так общие факторы, чтобы они как можно в большей мере вариацию исходных показателей. В модели делается предположение, что исходные показатели имеют среднее значение, равное 0 и дисперсию = 1. Это предположение не выполняется ни для одной переменной, поэтому предварительно данные нормируются.

Окружение scikit learn предоставляет програмные средства для подбора оптимального 
числа факторов, а так же обучаемую модель. 
    
    \begin{Verbatim}[commandchars=\\\{\}]
{\color{incolor}In [{\color{incolor}69}]:} \PY{k+kn}{import} \PY{n+nn}{numpy} \PY{k}{as} \PY{n+nn}{np}
         \PY{k+kn}{import} \PY{n+nn}{matplotlib}
         \PY{k+kn}{import} \PY{n+nn}{matplotlib}\PY{n+nn}{.}\PY{n+nn}{pyplot} \PY{k}{as} \PY{n+nn}{plt}
         \PY{k+kn}{from} \PY{n+nn}{sklearn}\PY{n+nn}{.}\PY{n+nn}{model\PYZus{}selection} \PY{k}{import} \PY{n}{cross\PYZus{}val\PYZus{}score}
         
         \PY{n}{n\PYZus{}features} \PY{o}{=} \PY{n+nb}{len}\PY{p}{(}\PY{n}{music}\PY{o}{.}\PY{n}{columns}\PY{p}{)}
         \PY{n}{n\PYZus{}components} \PY{o}{=} \PY{n}{np}\PY{o}{.}\PY{n}{arange}\PY{p}{(}\PY{l+m+mi}{0}\PY{p}{,} \PY{l+m+mi}{4}\PY{p}{)}
         \PY{n}{fa\PYZus{}scores} \PY{o}{=} \PY{p}{[}\PY{p}{]}
         \PY{n}{fa} \PY{o}{=} \PY{n}{FactorAnalysis}\PY{p}{(}\PY{p}{)}
         \PY{n}{fa}\PY{o}{.}\PY{n}{fit}\PY{p}{(}\PY{n}{music}\PY{p}{)}
         \PY{k}{for} \PY{n}{n} \PY{o+ow}{in} \PY{n}{n\PYZus{}components}\PY{p}{:}
             \PY{n}{fa}\PY{o}{.}\PY{n}{n\PYZus{}components} \PY{o}{=} \PY{n}{n}
             \PY{n}{fa\PYZus{}scores}\PY{o}{.}\PY{n}{append}\PY{p}{(}\PY{n}{np}\PY{o}{.}\PY{n}{mean}\PY{p}{(}\PY{n}{cross\PYZus{}val\PYZus{}score}\PY{p}{(}\PY{n}{fa}\PY{p}{,} \PY{n}{music}\PY{p}{)}\PY{p}{)}\PY{p}{)}
         \PY{n}{n\PYZus{}components\PYZus{}fa} \PY{o}{=} \PY{n}{n\PYZus{}components}\PY{p}{[}\PY{n}{np}\PY{o}{.}\PY{n}{argmax}\PY{p}{(}\PY{n}{fa\PYZus{}scores}\PY{p}{)}\PY{p}{]}
         \PY{n+nb}{print}\PY{p}{(}\PY{n}{n\PYZus{}components\PYZus{}fa}\PY{p}{)}
\end{Verbatim}


    \begin{Verbatim}[commandchars=\\\{\}]
3

    \end{Verbatim}

    \begin{Verbatim}[commandchars=\\\{\}]
{\color{incolor}In [{\color{incolor}72}]:} \PY{k+kn}{from} \PY{n+nn}{sklearn}\PY{n+nn}{.}\PY{n+nn}{decomposition} \PY{k}{import} \PY{n}{FactorAnalysis}
         \PY{n}{factor} \PY{o}{=} \PY{n}{FactorAnalysis}\PY{p}{(}\PY{n}{n\PYZus{}components}\PY{o}{=}\PY{l+m+mi}{3}\PY{p}{)}
         \PY{n}{factor}\PY{o}{.}\PY{n}{fit}\PY{p}{(}\PY{n}{music}\PY{p}{)}
         \PY{n+nb}{print} \PY{p}{(}\PY{p}{(}\PY{n}{pd}\PY{o}{.}\PY{n}{DataFrame}\PY{p}{(}\PY{n}{factor}\PY{o}{.}\PY{n}{components\PYZus{}}\PY{p}{,}\PY{n}{columns}\PY{o}{=}\PY{n}{music}\PY{o}{.}\PY{n}{columns}\PY{p}{)}\PY{p}{)}\PY{o}{.}\PY{n}{transpose}\PY{p}{(}\PY{p}{)}\PY{p}{)}
\end{Verbatim}


    \begin{Verbatim}[commandchars=\\\{\}]
                          0         1         2
Dance             -0.303149  0.665696 -0.425720
Folk               0.311782  0.402950  0.319220
Country            0.313558  0.329835  0.181988
Classical music    0.553627  0.341860  0.451575
Musical            0.262464  0.568232  0.329080
Pop               -0.271719  0.560852 -0.217120
Rock               0.801508 -0.061497 -0.227127
Metal or Hardrock  0.876026 -0.348435 -0.235428
Punk               0.816995 -0.247060 -0.472734
Hiphop, Rap       -0.374550  0.465859 -0.570731
Reggae, Ska        0.342663  0.340363 -0.349031
Swing, Jazz        0.604564  0.515972  0.191347
Rock n roll        0.780514  0.306515 -0.001735
Alternative        0.737268 -0.003769 -0.025753
Latino             0.043111  0.813650  0.124829
Techno, Trance    -0.234808  0.353415 -0.452851

    \end{Verbatim}

  \begin{Verbatim}[commandchars=\\\{\}]
                          0         1         2
Dance                        0.665696 
Folk                                   0.319220
Country                                
Classical music                        0.451575
Musical                      0.568232  0.329080
Pop                          0.560852 
Rock               0.801508           
Metal or Hardrock  0.876026            
Punk               0.816995            
Hiphop, Rap                            
Reggae, Ska                            
Swing, Jazz                            
Rock n roll        0.780514            
Alternative        0.737268             
Latino                       0.813650   
Techno, Trance                         

    \end{Verbatim}

\subsection{Вывод}
Данное разбиение дает три легко интерпетируемые группы: в первой 
любители тяжелой музыик, во вторую попали люди, любящие танцевальную
музыку, а третью -- любители классики.

\section{Классификация через логистическую регрессию}

\subsection{Постановка задачи}
Попробуем ответить на вопрос: что влияет на то, что человек бережливо относится к деньгам? 


\subsection{Предобработка данных}
    \begin{Verbatim}[commandchars=\\\{\}]
{\color{incolor}In [{\color{incolor}1}]:} \PY{k+kn}{import} \PY{n+nn}{pandas} \PY{k}{as} \PY{n+nn}{pd}
        \PY{k+kn}{import} \PY{n+nn}{numpy} \PY{k}{as} \PY{n+nn}{np}
        
        \PY{k+kn}{import} \PY{n+nn}{matplotlib}\PY{n+nn}{.}\PY{n+nn}{pyplot} \PY{k}{as} \PY{n+nn}{plt}
        \PY{k+kn}{import} \PY{n+nn}{seaborn} \PY{k}{as} \PY{n+nn}{sns}
        \PY{o}{\PYZpc{}}\PY{k}{matplotlib} inline
        \PY{n}{plt}\PY{o}{.}\PY{n}{style}\PY{o}{.}\PY{n}{use}\PY{p}{(}\PY{l+s+s1}{\PYZsq{}}\PY{l+s+s1}{fivethirtyeight}\PY{l+s+s1}{\PYZsq{}}\PY{p}{)}
\end{Verbatim}

Загрузим данные


    \begin{Verbatim}[commandchars=\\\{\}]
{\color{incolor}In [{\color{incolor}3}]:} 
        \PY{n}{df1} \PY{o}{=} \PY{n}{pd}\PY{o}{.}\PY{n}{read\PYZus{}csv}\PY{p}{(}\PY{l+s+s1}{\PYZsq{}}\PY{l+s+s1}{responses.csv}\PY{l+s+s1}{\PYZsq{}}\PY{p}{)}
\end{Verbatim}


 
Выделим подмножество значений, которые будем использовать в дальнейшем исследовании

            
    \begin{Verbatim}[commandchars=\\\{\}]
{\color{incolor}In [{\color{incolor}5}]:}\PY{n}{mov\PYZus{}mus}   \PY{o}{=} \PY{n}{df1}\PY{o}{.}\PY{n}{iloc}\PY{p}{[}\PY{p}{:}\PY{p}{,}\PY{p}{[}\PY{l+m+mi}{0}\PY{p}{,}\PY{l+m+mi}{19}\PY{p}{]}\PY{p}{]}
        \PY{n}{scared}    \PY{o}{=} \PY{n}{df1}\PY{o}{.}\PY{n}{iloc}\PY{p}{[}\PY{p}{:}\PY{p}{,}\PY{l+m+mi}{63}\PY{p}{:}\PY{l+m+mi}{73}\PY{p}{]}
        \PY{n}{interests} \PY{o}{=} \PY{n}{df1}\PY{o}{.}\PY{n}{iloc}\PY{p}{[}\PY{p}{:}\PY{p}{,}\PY{l+m+mi}{31}\PY{p}{:}\PY{l+m+mi}{63}\PY{p}{]}
        \PY{n}{demo}      \PY{o}{=} \PY{n}{df1}\PY{o}{.}\PY{n}{iloc}\PY{p}{[}\PY{p}{:}\PY{p}{,}\PY{l+m+mi}{140}\PY{p}{:}\PY{l+m+mi}{150}\PY{p}{]}
        \PY{n}{spending}  \PY{o}{=} \PY{n}{df1}\PY{o}{.}\PY{n}{iloc}\PY{p}{[}\PY{p}{:}\PY{p}{,}\PY{l+m+mi}{134}\PY{p}{:}\PY{l+m+mi}{140}\PY{p}{]}
        \PY{n}{predict}   \PY{o}{=} \PY{n}{df1}\PY{o}{.}\PY{n}{iloc}\PY{p}{[}\PY{p}{:}\PY{p}{,}\PY{l+m+mi}{133}\PY{p}{]}
        
        \PY{n}{scared}\PY{o}{.}\PY{n}{fillna}\PY{p}{(}\PY{l+m+mi}{0}\PY{p}{,} \PY{n}{inplace}\PY{o}{=}\PY{k+kc}{True}\PY{p}{)}
        \PY{n}{scared} \PY{o}{=} \PY{n}{scared}\PY{o}{.}\PY{n}{mean}\PY{p}{(}\PY{n}{axis}\PY{o}{=}\PY{l+m+mi}{1}\PY{p}{)}
        
        \PY{n}{df2} \PY{o}{=} \PY{n}{mov\PYZus{}mus}\PY{o}{.}\PY{n}{join}\PY{p}{(}\PY{p}{[}\PY{n}{scared}\PY{p}{,} \PY{n}{interests}\PY{p}{,} \PY{n}{demo}\PY{p}{,} \PY{n}{spending}\PY{p}{,} \PY{n}{predict}\PY{p}{]}\PY{p}{)}
        \PY{n}{df2}\PY{o}{.}\PY{n}{rename}\PY{p}{(}\PY{n}{columns}\PY{o}{=}\PY{p}{\PYZob{}}\PY{l+m+mi}{0}\PY{p}{:}\PY{l+s+s1}{\PYZsq{}}\PY{l+s+s1}{Scared}\PY{l+s+s1}{\PYZsq{}}\PY{p}{\PYZcb{}}\PY{p}{,} \PY{n}{inplace}\PY{o}{=}\PY{k+kc}{True}\PY{p}{)}
\end{Verbatim}


    \begin{Verbatim}[commandchars=\\\{\}]
{\color{incolor}In [{\color{incolor}6}]:} \PY{n}{df2}\PY{o}{.}\PY{n}{info}\PY{p}{(}\PY{p}{)}
\end{Verbatim}

Получили такое подмножество

    \begin{Verbatim}[commandchars=\\\{\}]
<class 'pandas.core.frame.DataFrame'>
RangeIndex: 1010 entries, 0 to 1009
Data columns (total 52 columns):
Music                         1007 non-null float64
Movies                        1004 non-null float64
Scared                        1010 non-null float64
History                       1008 non-null float64
Psychology                    1005 non-null float64
Politics                      1009 non-null float64
Mathematics                   1007 non-null float64
Physics                       1007 non-null float64
Internet                      1006 non-null float64
PC                            1004 non-null float64
Economy Management            1005 non-null float64
Biology                       1004 non-null float64
Chemistry                     1000 non-null float64
Reading                       1004 non-null float64
Geography                     1001 non-null float64
Foreign languages             1005 non-null float64
Medicine                      1005 non-null float64
Law                           1009 non-null float64
Cars                          1006 non-null float64
Art exhibitions               1004 non-null float64
Religion                      1007 non-null float64
Countryside, outdoors         1003 non-null float64
Dancing                       1007 non-null float64
Musical instruments           1009 non-null float64
Writing                       1004 non-null float64
Passive sport                 995 non-null float64
Active sport                  1006 non-null float64
Gardening                     1003 non-null float64
Celebrities                   1008 non-null float64
Shopping                      1008 non-null float64
Science and technology        1004 non-null float64
Theatre                       1002 non-null float64
Fun with friends              1006 non-null float64
Adrenaline sports             1007 non-null float64
Pets                          1006 non-null float64
Age                           1003 non-null float64
Height                        990 non-null float64
Weight                        990 non-null float64
Number of siblings            1004 non-null float64
Gender                        1004 non-null object
Left - right handed           1007 non-null object
Education                     1009 non-null object
Only child                    1008 non-null object
Village - town                1006 non-null object
House - block of flats        1006 non-null object
Shopping centres              1008 non-null float64
Branded clothing              1008 non-null float64
Entertainment spending        1007 non-null float64
Spending on looks             1007 non-null float64
Spending on gadgets           1010 non-null int64
Spending on healthy eating    1008 non-null float64
Finances                      1007 non-null float64
dtypes: float64(45), int64(1), object(6)
memory usage: 410.4+ KB

    \end{Verbatim}

Обработаем пропущенные данные 

    \begin{Verbatim}[commandchars=\\\{\}]
{\color{incolor}In [{\color{incolor}7}]:} 
        \PY{n}{drop\PYZus{}list} \PY{o}{=} \PY{p}{[}\PY{l+s+s1}{\PYZsq{}}\PY{l+s+s1}{Gender}\PY{l+s+s1}{\PYZsq{}}\PY{p}{,}\PY{l+s+s1}{\PYZsq{}}\PY{l+s+s1}{Left \PYZhy{} right handed}\PY{l+s+s1}{\PYZsq{}}\PY{p}{,}\PY{l+s+s1}{\PYZsq{}}\PY{l+s+s1}{Education}\PY{l+s+s1}{\PYZsq{}}\PY{p}{,}\PY{l+s+s1}{\PYZsq{}}\PY{l+s+s1}{Only child}\PY{l+s+s1}{\PYZsq{}}\PY{p}{,}\PY{l+s+s1}{\PYZsq{}}\PY{l+s+s1}{Village \PYZhy{} town}\PY{l+s+s1}{\PYZsq{}}\PY{p}{,}\PY{l+s+s1}{\PYZsq{}}\PY{l+s+s1}{House \PYZhy{} block of flats}\PY{l+s+s1}{\PYZsq{}}\PY{p}{,}\PY{l+s+s1}{\PYZsq{}}\PY{l+s+s1}{Finances}\PY{l+s+s1}{\PYZsq{}}\PY{p}{]}
        \PY{n}{df2}\PY{o}{.}\PY{n}{dropna}\PY{p}{(}\PY{n}{subset}\PY{o}{=} \PY{n}{drop\PYZus{}list}\PY{p}{,} \PY{n}{inplace}\PY{o}{=}\PY{k+kc}{True}\PY{p}{)}
        \PY{n}{df2}\PY{o}{.}\PY{n}{fillna}\PY{p}{(}\PY{l+m+mi}{0}\PY{p}{,} \PY{n}{inplace}\PY{o}{=}\PY{k+kc}{True}\PY{p}{)}
\end{Verbatim}

и представим наглядно те данные с которыми будем работать

    \begin{Verbatim}[commandchars=\\\{\}]
{\color{incolor}In [{\color{incolor}10}]:} \PY{n}{plt\PYZus{}dict} \PY{o}{=} \PY{p}{\PYZob{}}\PY{p}{\PYZcb{}}
         
         \PY{k}{for} \PY{n}{i} \PY{o+ow}{in} \PY{n+nb}{range}\PY{p}{(}\PY{l+m+mi}{0}\PY{p}{,}\PY{n+nb}{len}\PY{p}{(}\PY{n}{interests}\PY{o}{.}\PY{n}{columns}\PY{p}{)}\PY{p}{)}\PY{p}{:}
             \PY{n}{plt\PYZus{}dict}\PY{o}{.}\PY{n}{update}\PY{p}{(}\PY{p}{\PYZob{}}\PY{n}{i}\PY{p}{:}\PY{n}{interests}\PY{o}{.}\PY{n}{columns}\PY{p}{[}\PY{n}{i}\PY{p}{]}\PY{p}{\PYZcb{}}\PY{p}{)}
                     
         \PY{n}{fig}\PY{p}{,} \PY{n}{ax} \PY{o}{=} \PY{n}{plt}\PY{o}{.}\PY{n}{subplots}\PY{p}{(}\PY{l+m+mi}{4}\PY{p}{,}\PY{l+m+mi}{8}\PY{p}{,}\PY{n}{figsize}\PY{o}{=}\PY{p}{(}\PY{l+m+mi}{16}\PY{p}{,}\PY{l+m+mi}{16}\PY{p}{)}\PY{p}{,} \PY{n}{sharey}\PY{o}{=}\PY{k+kc}{True}\PY{p}{,} \PY{n}{sharex}\PY{o}{=}\PY{k+kc}{True}\PY{p}{)}
         
         \PY{n}{initial} \PY{o}{=} \PY{l+m+mi}{0}
         
         \PY{k}{for} \PY{n}{i} \PY{o+ow}{in} \PY{n+nb}{range}\PY{p}{(}\PY{l+m+mi}{4}\PY{p}{)}\PY{p}{:}
             \PY{k}{for} \PY{n}{j} \PY{o+ow}{in} \PY{n+nb}{range}\PY{p}{(}\PY{l+m+mi}{8}\PY{p}{)}\PY{p}{:}
                 \PY{n}{sns}\PY{o}{.}\PY{n}{countplot}\PY{p}{(}\PY{n}{df2}\PY{p}{[}\PY{n}{plt\PYZus{}dict}\PY{p}{[}\PY{n}{initial}\PY{p}{]}\PY{p}{]}\PY{p}{,} \PY{n}{ax}\PY{o}{=}\PY{n}{ax}\PY{p}{[}\PY{n}{i}\PY{p}{,}\PY{n}{j}\PY{p}{]}\PY{p}{)}
                 \PY{n}{ax}\PY{p}{[}\PY{n}{i}\PY{p}{,}\PY{n}{j}\PY{p}{]}\PY{o}{.}\PY{n}{set\PYZus{}ylabel}\PY{p}{(}\PY{l+s+s1}{\PYZsq{}}\PY{l+s+s1}{\PYZsq{}}\PY{p}{)}
                 \PY{n}{ax}\PY{p}{[}\PY{n}{i}\PY{p}{,}\PY{n}{j}\PY{p}{]}\PY{o}{.}\PY{n}{set\PYZus{}xlabel}\PY{p}{(}\PY{l+s+s1}{\PYZsq{}}\PY{l+s+s1}{\PYZsq{}}\PY{p}{)}
                 \PY{n}{ax}\PY{p}{[}\PY{n}{i}\PY{p}{,}\PY{n}{j}\PY{p}{]}\PY{o}{.}\PY{n}{set\PYZus{}xticklabels}\PY{p}{(}\PY{n}{labels}\PY{o}{=}\PY{n}{np}\PY{o}{.}\PY{n}{arange}\PY{p}{(}\PY{l+m+mi}{0}\PY{p}{,}\PY{l+m+mi}{6}\PY{p}{)}\PY{p}{,} \PY{n}{fontsize}\PY{o}{=}\PY{l+m+mi}{5}\PY{p}{)}
                 \PY{n}{ax}\PY{p}{[}\PY{n}{i}\PY{p}{,}\PY{n}{j}\PY{p}{]}\PY{o}{.}\PY{n}{set\PYZus{}yticklabels}\PY{p}{(}\PY{n}{labels}\PY{o}{=}\PY{n}{np}\PY{o}{.}\PY{n}{arange}\PY{p}{(}\PY{l+m+mi}{0}\PY{p}{,}\PY{l+m+mi}{601}\PY{p}{,}\PY{l+m+mi}{100}\PY{p}{)}\PY{p}{,} \PY{n}{fontsize}\PY{o}{=}\PY{l+m+mi}{5}\PY{p}{)}
                 \PY{n}{ax}\PY{p}{[}\PY{n}{i}\PY{p}{,}\PY{n}{j}\PY{p}{]}\PY{o}{.}\PY{n}{set\PYZus{}title}\PY{p}{(}\PY{n}{plt\PYZus{}dict}\PY{p}{[}\PY{n}{initial}\PY{p}{]}\PY{p}{,} \PY{n}{fontsize}\PY{o}{=}\PY{l+m+mi}{10}\PY{p}{)}
                 \PY{n}{ax}\PY{p}{[}\PY{n}{i}\PY{p}{,}\PY{n}{j}\PY{p}{]}\PY{o}{.}\PY{n}{set\PYZus{}xlim}\PY{p}{(}\PY{o}{.}\PY{l+m+mi}{5}\PY{p}{,}\PY{l+m+mf}{5.5}\PY{p}{)}
                 \PY{n}{ax}\PY{p}{[}\PY{n}{i}\PY{p}{,}\PY{n}{j}\PY{p}{]}\PY{o}{.}\PY{n}{set\PYZus{}ylim}\PY{p}{(}\PY{l+m+mi}{0}\PY{p}{,}\PY{l+m+mi}{600}\PY{p}{)}
                 \PY{n}{initial} \PY{o}{+}\PY{o}{=} \PY{l+m+mi}{1}
\end{Verbatim}


    \begin{center}
    \adjustimage{max size={0.9\linewidth}{0.9\paperheight}}{output_6_0.png}
    \end{center}
    { \hspace*{\fill} \\}
    
Часть данных представленна номинальными переменными. Выведем их 
    
    \begin{Verbatim}[commandchars=\\\{\}]
{\color{incolor}In [{\color{incolor}13}]:} \PY{n}{obj\PYZus{}dict} \PY{o}{=} \PY{p}{\PYZob{}}\PY{l+m+mi}{0}\PY{p}{:}\PY{l+s+s1}{\PYZsq{}}\PY{l+s+s1}{Gender}\PY{l+s+s1}{\PYZsq{}}\PY{p}{,} \PY{l+m+mi}{1}\PY{p}{:}\PY{l+s+s1}{\PYZsq{}}\PY{l+s+s1}{Left \PYZhy{} right handed}\PY{l+s+s1}{\PYZsq{}}\PY{p}{,} \PY{l+m+mi}{2}\PY{p}{:}\PY{l+s+s1}{\PYZsq{}}\PY{l+s+s1}{Only child}\PY{l+s+s1}{\PYZsq{}}\PY{p}{,} \PY{l+m+mi}{3}\PY{p}{:}\PY{l+s+s1}{\PYZsq{}}\PY{l+s+s1}{Village \PYZhy{} town}\PY{l+s+s1}{\PYZsq{}}\PY{p}{,} \PY{l+m+mi}{4}\PY{p}{:}\PY{l+s+s1}{\PYZsq{}}\PY{l+s+s1}{House \PYZhy{} block of flats}\PY{l+s+s1}{\PYZsq{}}\PY{p}{,} \PY{l+m+mi}{5}\PY{p}{:}\PY{l+s+s1}{\PYZsq{}}\PY{l+s+s1}{Education}\PY{l+s+s1}{\PYZsq{}}\PY{p}{\PYZcb{}}
         \PY{n}{fig}\PY{p}{,} \PY{n}{ax} \PY{o}{=} \PY{n}{plt}\PY{o}{.}\PY{n}{subplots}\PY{p}{(}\PY{l+m+mi}{2}\PY{p}{,}\PY{l+m+mi}{3}\PY{p}{,} \PY{n}{figsize}\PY{o}{=}\PY{p}{(}\PY{l+m+mi}{16}\PY{p}{,}\PY{l+m+mi}{16}\PY{p}{)}\PY{p}{,} \PY{n}{sharey}\PY{o}{=}\PY{k+kc}{True}\PY{p}{)}
         
         \PY{n}{initial} \PY{o}{=} \PY{l+m+mi}{0}
         
         \PY{k}{for} \PY{n}{i} \PY{o+ow}{in} \PY{n+nb}{range}\PY{p}{(}\PY{l+m+mi}{2}\PY{p}{)}\PY{p}{:}
             \PY{k}{for} \PY{n}{j} \PY{o+ow}{in} \PY{n+nb}{range}\PY{p}{(}\PY{l+m+mi}{3}\PY{p}{)}\PY{p}{:}
                 \PY{n}{sns}\PY{o}{.}\PY{n}{countplot}\PY{p}{(}\PY{n}{df2}\PY{p}{[}\PY{n}{obj\PYZus{}dict}\PY{p}{[}\PY{n}{initial}\PY{p}{]}\PY{p}{]}\PY{p}{,} \PY{n}{ax}\PY{o}{=}\PY{n}{ax}\PY{p}{[}\PY{n}{i}\PY{p}{,}\PY{n}{j}\PY{p}{]}\PY{p}{)}
                 \PY{n}{ax}\PY{p}{[}\PY{n}{i}\PY{p}{,}\PY{n}{j}\PY{p}{]}\PY{o}{.}\PY{n}{set\PYZus{}title}\PY{p}{(}\PY{n}{obj\PYZus{}dict}\PY{p}{[}\PY{n}{initial}\PY{p}{]}\PY{p}{,} \PY{n}{fontsize}\PY{o}{=}\PY{l+m+mi}{15}\PY{p}{)}
                 \PY{n}{ax}\PY{p}{[}\PY{n}{i}\PY{p}{,}\PY{n}{j}\PY{p}{]}\PY{o}{.}\PY{n}{set\PYZus{}xlabel}\PY{p}{(}\PY{l+s+s1}{\PYZsq{}}\PY{l+s+s1}{\PYZsq{}}\PY{p}{)}
                 \PY{n}{ax}\PY{p}{[}\PY{n}{i}\PY{p}{,}\PY{n}{j}\PY{p}{]}\PY{o}{.}\PY{n}{set\PYZus{}ylabel}\PY{p}{(}\PY{l+s+s1}{\PYZsq{}}\PY{l+s+s1}{\PYZsq{}}\PY{p}{)}
                 \PY{n}{ax}\PY{p}{[}\PY{n}{i}\PY{p}{,}\PY{n}{j}\PY{p}{]}\PY{o}{.}\PY{n}{set\PYZus{}xticklabels}\PY{p}{(}\PY{n}{labels}\PY{o}{=}\PY{n}{df2}\PY{p}{[}\PY{n}{obj\PYZus{}dict}\PY{p}{[}\PY{n}{initial}\PY{p}{]}\PY{p}{]}\PY{o}{.}\PY{n}{unique}\PY{p}{(}\PY{p}{)}\PY{p}{,} \PY{n}{fontsize}\PY{o}{=}\PY{l+m+mi}{12}\PY{p}{)}
                 \PY{n}{initial} \PY{o}{+}\PY{o}{=} \PY{l+m+mi}{1}
         
         \PY{n}{ax}\PY{p}{[}\PY{l+m+mi}{1}\PY{p}{,}\PY{l+m+mi}{2}\PY{p}{]}\PY{o}{.}\PY{n}{set\PYZus{}xticklabels}\PY{p}{(}\PY{n}{labels}\PY{o}{=}\PY{n}{df2}\PY{p}{[}\PY{l+s+s1}{\PYZsq{}}\PY{l+s+s1}{Education}\PY{l+s+s1}{\PYZsq{}}\PY{p}{]}\PY{o}{.}\PY{n}{unique}\PY{p}{(}\PY{p}{)}\PY{p}{,} \PY{n}{rotation}\PY{o}{=}\PY{l+m+mi}{20}\PY{p}{,} \PY{n}{fontsize}\PY{o}{=}\PY{l+m+mi}{9}\PY{p}{)}
\end{Verbatim}


\begin{Verbatim}[commandchars=\\\{\}]
{\color{outcolor}Out[{\color{outcolor}13}]:} [<matplotlib.text.Text at 0x7f3d38ecfd30>,
          <matplotlib.text.Text at 0x7f3d38aa6588>,
          <matplotlib.text.Text at 0x7f3d3917bcc0>,
          <matplotlib.text.Text at 0x7f3d3880bf60>,
          <matplotlib.text.Text at 0x7f3d3938f550>,
          <matplotlib.text.Text at 0x7f3d38ae7ef0>]
\end{Verbatim}
            
    \begin{center}
    \adjustimage{max size={0.9\linewidth}{0.9\paperheight}}{output_7_1.png}
    \end{center}
    { \hspace*{\fill} \\}

и преобразуем к целочисленным переменным, 
чтобы их можно было обрабатывать        
    
    \begin{Verbatim}[commandchars=\\\{\}]
{\color{incolor}In [{\color{incolor}14}]:} \PY{n}{gender}  \PY{o}{=} \PY{n}{pd}\PY{o}{.}\PY{n}{get\PYZus{}dummies}\PY{p}{(}\PY{n}{df2}\PY{p}{[}\PY{l+s+s1}{\PYZsq{}}\PY{l+s+s1}{Gender}\PY{l+s+s1}{\PYZsq{}}\PY{p}{]}\PY{p}{)}
         \PY{n}{handed}  \PY{o}{=} \PY{n}{pd}\PY{o}{.}\PY{n}{get\PYZus{}dummies}\PY{p}{(}\PY{n}{df2}\PY{p}{[}\PY{l+s+s1}{\PYZsq{}}\PY{l+s+s1}{Left \PYZhy{} right handed}\PY{l+s+s1}{\PYZsq{}}\PY{p}{]}\PY{p}{)}
         \PY{n}{child}   \PY{o}{=} \PY{n}{pd}\PY{o}{.}\PY{n}{get\PYZus{}dummies}\PY{p}{(}\PY{n}{df2}\PY{p}{[}\PY{l+s+s1}{\PYZsq{}}\PY{l+s+s1}{Only child}\PY{l+s+s1}{\PYZsq{}}\PY{p}{]}\PY{p}{)}
         \PY{n}{vil\PYZus{}tow} \PY{o}{=} \PY{n}{pd}\PY{o}{.}\PY{n}{get\PYZus{}dummies}\PY{p}{(}\PY{n}{df2}\PY{p}{[}\PY{l+s+s1}{\PYZsq{}}\PY{l+s+s1}{Village \PYZhy{} town}\PY{l+s+s1}{\PYZsq{}}\PY{p}{]}\PY{p}{)}
         \PY{n}{resid}   \PY{o}{=} \PY{n}{pd}\PY{o}{.}\PY{n}{get\PYZus{}dummies}\PY{p}{(}\PY{n}{df2}\PY{p}{[}\PY{l+s+s1}{\PYZsq{}}\PY{l+s+s1}{House \PYZhy{} block of flats}\PY{l+s+s1}{\PYZsq{}}\PY{p}{]}\PY{p}{)}
         \PY{n}{educa}   \PY{o}{=} \PY{n}{pd}\PY{o}{.}\PY{n}{get\PYZus{}dummies}\PY{p}{(}\PY{n}{df2}\PY{p}{[}\PY{l+s+s1}{\PYZsq{}}\PY{l+s+s1}{Education}\PY{l+s+s1}{\PYZsq{}}\PY{p}{]}\PY{p}{)}
         
         \PY{n}{df2}\PY{o}{.}\PY{n}{drop}\PY{p}{(}\PY{p}{[}\PY{l+s+s1}{\PYZsq{}}\PY{l+s+s1}{Gender}\PY{l+s+s1}{\PYZsq{}}\PY{p}{,}\PY{l+s+s1}{\PYZsq{}}\PY{l+s+s1}{Left \PYZhy{} right handed}\PY{l+s+s1}{\PYZsq{}}\PY{p}{,}\PY{l+s+s1}{\PYZsq{}}\PY{l+s+s1}{Only child}\PY{l+s+s1}{\PYZsq{}}\PY{p}{,}\PY{l+s+s1}{\PYZsq{}}\PY{l+s+s1}{Village \PYZhy{} town}\PY{l+s+s1}{\PYZsq{}}\PY{p}{,}\PY{l+s+s1}{\PYZsq{}}\PY{l+s+s1}{House \PYZhy{} block of flats}\PY{l+s+s1}{\PYZsq{}}\PY{p}{,}\PY{l+s+s1}{\PYZsq{}}\PY{l+s+s1}{Education}\PY{l+s+s1}{\PYZsq{}}\PY{p}{]}\PY{p}{,} \PY{n}{axis}\PY{o}{=}\PY{l+m+mi}{1}\PY{p}{,} \PY{n}{inplace}\PY{o}{=}\PY{k+kc}{True}\PY{p}{)}
         \PY{n}{df2} \PY{o}{=} \PY{n}{df2}\PY{o}{.}\PY{n}{join}\PY{p}{(}\PY{p}{[}\PY{n}{gender}\PY{p}{,} \PY{n}{handed}\PY{p}{,} \PY{n}{child}\PY{p}{,} \PY{n}{vil\PYZus{}tow}\PY{p}{,} \PY{n}{resid}\PY{p}{,} \PY{n}{educa}\PY{p}{]}\PY{p}{)}
\end{Verbatim}


    \begin{Verbatim}[commandchars=\\\{\}]
{\color{incolor}In [{\color{incolor}27}]:} \PY{c+c1}{\PYZsh{}отобразим характер расходовы по категориям}
         \PY{n}{spend\PYZus{}aves} \PY{o}{=} \PY{n}{pd}\PY{o}{.}\PY{n}{Series}\PY{p}{(}\PY{n}{df2}\PY{p}{[}\PY{n}{spending}\PY{o}{.}\PY{n}{columns}\PY{p}{]}\PY{o}{.}\PY{n}{mean}\PY{p}{(}\PY{p}{)}\PY{p}{)}
         \PY{n}{spend\PYZus{}aves} \PY{o}{=} \PY{n}{spend\PYZus{}aves}\PY{o}{.}\PY{n}{append}\PY{p}{(}\PY{n}{pd}\PY{o}{.}\PY{n}{Series}\PY{p}{(}\PY{n}{df2}\PY{p}{[}\PY{l+s+s1}{\PYZsq{}}\PY{l+s+s1}{Finances}\PY{l+s+s1}{\PYZsq{}}\PY{p}{]}\PY{o}{.}\PY{n}{mean}\PY{p}{(}\PY{p}{)}\PY{p}{,} \PY{n}{index}\PY{o}{=}\PY{p}{[}\PY{l+s+s1}{\PYZsq{}}\PY{l+s+s1}{Finances}\PY{l+s+s1}{\PYZsq{}}\PY{p}{]}\PY{p}{)}\PY{p}{)}
         \PY{n}{spend\PYZus{}aves}\PY{o}{.}\PY{n}{plot}\PY{p}{(}
             \PY{n}{figsize}\PY{o}{=}\PY{p}{(}\PY{l+m+mi}{6}\PY{p}{,}\PY{l+m+mi}{5}\PY{p}{)}\PY{p}{,} \PY{n}{kind}\PY{o}{=}\PY{l+s+s1}{\PYZsq{}}\PY{l+s+s1}{barh}\PY{l+s+s1}{\PYZsq{}}\PY{p}{,} \PY{n}{title}\PY{o}{=}\PY{l+s+s1}{\PYZsq{}}\PY{l+s+s1}{Средний ответ}\PY{l+s+s1}{\PYZsq{}}\PY{p}{,} 
             \PY{n}{color}\PY{o}{=}\PY{p}{[}\PY{l+s+s2}{\PYZdq{}}\PY{l+s+s2}{\PYZsh{}30a2da}\PY{l+s+s2}{\PYZdq{}}\PY{p}{,}\PY{l+s+s2}{\PYZdq{}}\PY{l+s+s2}{\PYZsh{}fc4f30}\PY{l+s+s2}{\PYZdq{}}\PY{p}{,}\PY{l+s+s2}{\PYZdq{}}\PY{l+s+s2}{\PYZsh{}e5ae38}\PY{l+s+s2}{\PYZdq{}}\PY{p}{,}\PY{l+s+s2}{\PYZdq{}}\PY{l+s+s2}{\PYZsh{}6d904f}\PY{l+s+s2}{\PYZdq{}}\PY{p}{,}\PY{l+s+s2}{\PYZdq{}}\PY{l+s+s2}{\PYZsh{}8b8b8b}\PY{l+s+s2}{\PYZdq{}}\PY{p}{,}\PY{l+s+s1}{\PYZsq{}}\PY{l+s+s1}{m}\PY{l+s+s1}{\PYZsq{}}\PY{p}{,} \PY{l+s+s1}{\PYZsq{}}\PY{l+s+s1}{r}\PY{l+s+s1}{\PYZsq{}}\PY{p}{]}\PY{p}{,} \PY{n}{xlim}\PY{o}{=}\PY{p}{(}\PY{l+m+mi}{1}\PY{p}{,}\PY{l+m+mi}{5}\PY{p}{)}\PY{p}{)}
         
         \PY{n}{plt}\PY{o}{.}\PY{n}{axvline}\PY{p}{(}\PY{n}{x}\PY{o}{=}\PY{n}{np}\PY{o}{.}\PY{n}{mean}\PY{p}{(}\PY{n}{spend\PYZus{}aves}\PY{p}{)}\PY{p}{,} \PY{n}{color}\PY{o}{=}\PY{l+s+s1}{\PYZsq{}}\PY{l+s+s1}{k}\PY{l+s+s1}{\PYZsq{}}\PY{p}{,} \PY{n}{lw}\PY{o}{=}\PY{l+m+mi}{4}\PY{p}{,} \PY{n}{ls}\PY{o}{=}\PY{l+s+s1}{\PYZsq{}}\PY{l+s+s1}{dashed}\PY{l+s+s1}{\PYZsq{}}\PY{p}{)}
         
         \PY{n+nb}{print}\PY{p}{(}\PY{l+s+s1}{\PYZsq{}}\PY{l+s+s1}{Студенты оценили свою способность сберегать в }\PY{l+s+si}{\PYZob{}:.2f\PYZcb{}}\PY{l+s+s1}{ \PYZhy{} но данные по всей таблице выше }\PY{l+s+si}{\PYZob{}:.2f\PYZcb{}}\PY{l+s+s1}{. }\PY{l+s+se}{\PYZbs{}n}\PY{l+s+s1}{ Это говорит о том, что молодежь тратит денег больше, чем думает}\PY{l+s+s1}{\PYZsq{}}
               \PY{o}{.}\PY{n}{format}\PY{p}{(}\PY{n}{df2}\PY{p}{[}\PY{l+s+s1}{\PYZsq{}}\PY{l+s+s1}{Finances}\PY{l+s+s1}{\PYZsq{}}\PY{p}{]}\PY{o}{.}\PY{n}{mean}\PY{p}{(}\PY{p}{)}\PY{p}{,}\PY{n}{np}\PY{o}{.}\PY{n}{mean}\PY{p}{(}\PY{n}{spend\PYZus{}aves}\PY{p}{)}\PY{p}{)}\PY{p}{)}
\end{Verbatim}


    \begin{Verbatim}[commandchars=\\\{\}]
Студенты оценили свою способность сберегать в 3.03 - но данные по всей таблице выше 3.15. 
 Это говорит о том, что молодежь тратит денег больше, чем думает, что тратит

    \end{Verbatim}

    \begin{center}
    \adjustimage{max size={0.9\linewidth}{0.9\paperheight}}{output_9_1.png}
    \end{center}
    { \hspace*{\fill} \\}
    
Чтобы не проводить многозначное прогнозирование, разобьем характеристику "финансы" на 2 группы 
-- 3 и меньше, и 4 и больше.
    
    \begin{Verbatim}[commandchars=\\\{\}]
{\color{incolor}In [{\color{incolor}28}]:}  \PY{n}{df2}\PY{o}{.}\PY{n}{loc}\PY{p}{[}\PY{n}{df2}\PY{p}{[}\PY{l+s+s1}{\PYZsq{}}\PY{l+s+s1}{Finances}\PY{l+s+s1}{\PYZsq{}}\PY{p}{]} \PY{o}{\PYZlt{}}\PY{o}{=} \PY{l+m+mi}{3}\PY{p}{,} \PY{l+s+s1}{\PYZsq{}}\PY{l+s+s1}{Finances}\PY{l+s+s1}{\PYZsq{}}\PY{p}{]} \PY{o}{=} \PY{l+m+mi}{0}
         \PY{n}{df2}\PY{o}{.}\PY{n}{loc}\PY{p}{[}\PY{n}{df2}\PY{p}{[}\PY{l+s+s1}{\PYZsq{}}\PY{l+s+s1}{Finances}\PY{l+s+s1}{\PYZsq{}}\PY{p}{]} \PY{o}{\PYZgt{}} \PY{l+m+mi}{3}\PY{p}{,} \PY{l+s+s1}{\PYZsq{}}\PY{l+s+s1}{Finances}\PY{l+s+s1}{\PYZsq{}}\PY{p}{]} \PY{o}{=} \PY{l+m+mi}{1}
\end{Verbatim}

\subsection{Идея метода}
Воспользуемся окружением scikit learn
kFold позволяет разделить данные 2 две пары выборок -- одну обучающую и вторую контрольную.
Для этого подготовим данные, перенеся столбец "финансы" в другую переменную.
Класс GridSearchCV позволяет определить оптимальный параметр для логистической регрессии.
Затем обучим нашу модель и сравним средную точность на тренировочной выборке и на тестовой.
Полученная точность не является идеальной, так как является "платой" за простоту модели.
Затем отобразим полученные важности характеристик, и, для удобства, выведем первые 10. 

\subsection{Реализация}
    \begin{Verbatim}[commandchars=\\\{\}]
{\color{incolor}In [{\color{incolor}29}]:} \PY{c+c1}{\PYZsh{}ML}
         \PY{k+kn}{from} \PY{n+nn}{sklearn}\PY{n+nn}{.}\PY{n+nn}{cross\PYZus{}validation} \PY{k}{import} \PY{n}{KFold}\PY{p}{,} \PY{n}{train\PYZus{}test\PYZus{}split}\PY{p}{,} \PY{n}{cross\PYZus{}val\PYZus{}score}
         \PY{k+kn}{from} \PY{n+nn}{sklearn}\PY{n+nn}{.}\PY{n+nn}{model\PYZus{}selection} \PY{k}{import} \PY{n}{GridSearchCV}
         \PY{k+kn}{from} \PY{n+nn}{sklearn}\PY{n+nn}{.}\PY{n+nn}{linear\PYZus{}model} \PY{k}{import} \PY{n}{LogisticRegression}
\end{Verbatim}


    \begin{Verbatim}[commandchars=\\\{\}]
/usr/local/lib/python3.4/dist-packages/sklearn/cross\_validation.py:44: DeprecationWarning: This module was deprecated in version 0.18 in favor of the model\_selection module into which all the refactored classes and functions are moved. Also note that the interface of the new CV iterators are different from that of this module. This module will be removed in 0.20.
  "This module will be removed in 0.20.", DeprecationWarning)

    \end{Verbatim}

    \begin{Verbatim}[commandchars=\\\{\}]
{\color{incolor}In [{\color{incolor}30}]:} \PY{c+c1}{\PYZsh{}set up data for modeling}
         \PY{n}{x} \PY{o}{=} \PY{n}{df2}\PY{o}{.}\PY{n}{drop}\PY{p}{(}\PY{l+s+s1}{\PYZsq{}}\PY{l+s+s1}{Finances}\PY{l+s+s1}{\PYZsq{}}\PY{p}{,} \PY{n}{axis}\PY{o}{=}\PY{l+m+mi}{1}\PY{p}{)}
         \PY{n}{y} \PY{o}{=} \PY{n}{df2}\PY{p}{[}\PY{l+s+s1}{\PYZsq{}}\PY{l+s+s1}{Finances}\PY{l+s+s1}{\PYZsq{}}\PY{p}{]}
         
         \PY{n}{x\PYZus{}train}\PY{p}{,} \PY{n}{x\PYZus{}test}\PY{p}{,} \PY{n}{y\PYZus{}train}\PY{p}{,} \PY{n}{y\PYZus{}test} \PY{o}{=} \PY{n}{train\PYZus{}test\PYZus{}split}\PY{p}{(}\PY{n}{x}\PY{p}{,} \PY{n}{y}\PY{p}{,} \PY{n}{test\PYZus{}size}\PY{o}{=}\PY{o}{.}\PY{l+m+mi}{3}\PY{p}{)}
         
         \PY{n}{kf} \PY{o}{=} \PY{n}{KFold}\PY{p}{(}\PY{n+nb}{len}\PY{p}{(}\PY{n}{x\PYZus{}train}\PY{p}{)}\PY{p}{,} \PY{n}{n\PYZus{}folds}\PY{o}{=}\PY{l+m+mi}{5}\PY{p}{)}
\end{Verbatim}


    \begin{Verbatim}[commandchars=\\\{\}]
{\color{incolor}In [{\color{incolor}31}]:} \PY{c+c1}{\PYZsh{}Use GridSearchCV for parameter tuning}
         \PY{n}{logreg} \PY{o}{=} \PY{n}{LogisticRegression}\PY{p}{(}\PY{p}{)}
         
         \PY{n}{param\PYZus{}grid} \PY{o}{=} \PY{p}{\PYZob{}}\PY{l+s+s1}{\PYZsq{}}\PY{l+s+s1}{C}\PY{l+s+s1}{\PYZsq{}}\PY{p}{:}\PY{p}{[}\PY{o}{.}\PY{l+m+mi}{01}\PY{p}{,}\PY{o}{.}\PY{l+m+mi}{03}\PY{p}{,}\PY{o}{.}\PY{l+m+mi}{1}\PY{p}{,}\PY{o}{.}\PY{l+m+mi}{3}\PY{p}{,}\PY{l+m+mi}{1}\PY{p}{,}\PY{l+m+mi}{3}\PY{p}{,}\PY{l+m+mi}{10}\PY{p}{]}\PY{p}{\PYZcb{}}
         
         \PY{n}{gs\PYZus{}logreg} \PY{o}{=} \PY{n}{GridSearchCV}\PY{p}{(}\PY{n}{logreg}\PY{p}{,} \PY{n}{param\PYZus{}grid}\PY{o}{=}\PY{n}{param\PYZus{}grid}\PY{p}{,} \PY{n}{cv}\PY{o}{=}\PY{n}{kf}\PY{p}{)}
         
         \PY{n}{gs\PYZus{}logreg}\PY{o}{.}\PY{n}{fit}\PY{p}{(}\PY{n}{x\PYZus{}train}\PY{p}{,} \PY{n}{y\PYZus{}train}\PY{p}{)}
         \PY{n}{gs\PYZus{}logreg}\PY{o}{.}\PY{n}{best\PYZus{}params\PYZus{}}
\end{Verbatim}


\begin{Verbatim}[commandchars=\\\{\}]
{\color{outcolor}Out[{\color{outcolor}31}]:} \{'C': 0.01\}
\end{Verbatim}
            
    \begin{Verbatim}[commandchars=\\\{\}]
{\color{incolor}In [{\color{incolor}32}]:} \PY{c+c1}{\PYZsh{}fit Logistic Regression model, eval scoring}
         \PY{n}{logreg} \PY{o}{=} \PY{n}{LogisticRegression}\PY{p}{(}\PY{n}{C}\PY{o}{=}\PY{o}{.}\PY{l+m+mi}{01}\PY{p}{)}
         \PY{n}{logreg}\PY{o}{.}\PY{n}{fit}\PY{p}{(}\PY{n}{x\PYZus{}train}\PY{p}{,} \PY{n}{y\PYZus{}train}\PY{p}{)}
         
         \PY{n+nb}{print}\PY{p}{(}\PY{l+s+s1}{\PYZsq{}}\PY{l+s+s1}{Average accuracy score on cv (KFold) set: }\PY{l+s+si}{\PYZob{}:.3f\PYZcb{}}\PY{l+s+s1}{\PYZsq{}}\PY{o}{.}\PY{n}{format}\PY{p}{(}\PY{n}{np}\PY{o}{.}\PY{n}{mean}\PY{p}{(}\PY{n}{cross\PYZus{}val\PYZus{}score}\PY{p}{(}\PY{n}{logreg}\PY{p}{,} \PY{n}{x\PYZus{}train}\PY{p}{,} \PY{n}{y\PYZus{}train}\PY{p}{,} \PY{n}{cv}\PY{o}{=}\PY{n}{kf}\PY{p}{)}\PY{p}{)}\PY{p}{)}\PY{p}{)}
         \PY{n+nb}{print}\PY{p}{(}\PY{l+s+s1}{\PYZsq{}}\PY{l+s+s1}{Accuracy score on test set is: }\PY{l+s+si}{\PYZob{}:.3f\PYZcb{}}\PY{l+s+s1}{\PYZsq{}}\PY{o}{.}\PY{n}{format}\PY{p}{(}\PY{n}{logreg}\PY{o}{.}\PY{n}{score}\PY{p}{(}\PY{n}{x\PYZus{}test}\PY{p}{,} \PY{n}{y\PYZus{}test}\PY{p}{)}\PY{p}{)}\PY{p}{)}
\end{Verbatim}


    \begin{Verbatim}[commandchars=\\\{\}]
Average accuracy score on cv (KFold) set: 0.638
Accuracy score on test set is: 0.670

    \end{Verbatim}

    \begin{Verbatim}[commandchars=\\\{\}]
{\color{incolor}In [{\color{incolor}34}]:} \PY{c+c1}{\PYZsh{}plot feature importance}
         \PY{n}{coeff\PYZus{}df} \PY{o}{=} \PY{n}{pd}\PY{o}{.}\PY{n}{DataFrame}\PY{p}{(}\PY{n}{data}\PY{o}{=}\PY{n}{logreg}\PY{o}{.}\PY{n}{coef\PYZus{}}\PY{p}{[}\PY{l+m+mi}{0}\PY{p}{]}\PY{p}{,} \PY{n}{index}\PY{o}{=}\PY{p}{[}\PY{n}{x\PYZus{}train}\PY{o}{.}\PY{n}{columns}\PY{p}{]}\PY{p}{,} \PY{n}{columns}\PY{o}{=}\PY{p}{[}\PY{l+s+s1}{\PYZsq{}}\PY{l+s+s1}{Feature\PYZus{}Import}\PY{l+s+s1}{\PYZsq{}}\PY{p}{]}\PY{p}{)}
         \PY{n}{coeff\PYZus{}df} \PY{o}{=} \PY{n}{coeff\PYZus{}df}\PY{o}{.}\PY{n}{sort\PYZus{}values}\PY{p}{(}\PY{n}{by}\PY{o}{=}\PY{l+s+s1}{\PYZsq{}}\PY{l+s+s1}{Feature\PYZus{}Import}\PY{l+s+s1}{\PYZsq{}}\PY{p}{,} \PY{n}{ascending}\PY{o}{=}\PY{k+kc}{False}\PY{p}{)}
         
         \PY{n}{fig}\PY{p}{,} \PY{n}{ax1} \PY{o}{=} \PY{n}{plt}\PY{o}{.}\PY{n}{subplots}\PY{p}{(}\PY{l+m+mi}{1}\PY{p}{,}\PY{l+m+mi}{1}\PY{p}{,} \PY{n}{figsize}\PY{o}{=}\PY{p}{(}\PY{l+m+mi}{16}\PY{p}{,}\PY{l+m+mi}{16}\PY{p}{)}\PY{p}{)}
         
         \PY{n}{sns}\PY{o}{.}\PY{n}{barplot}\PY{p}{(}\PY{n}{x}\PY{o}{=}\PY{n}{coeff\PYZus{}df}\PY{o}{.}\PY{n}{index}\PY{p}{,} \PY{n}{y}\PY{o}{=}\PY{n}{coeff\PYZus{}df}\PY{p}{[}\PY{l+s+s1}{\PYZsq{}}\PY{l+s+s1}{Feature\PYZus{}Import}\PY{l+s+s1}{\PYZsq{}}\PY{p}{]}\PY{p}{,} \PY{n}{ax}\PY{o}{=}\PY{n}{ax1}\PY{p}{)}
         \PY{n}{ax1}\PY{o}{.}\PY{n}{set\PYZus{}title}\PY{p}{(}\PY{l+s+s1}{\PYZsq{}}\PY{l+s+s1}{All Features}\PY{l+s+s1}{\PYZsq{}}\PY{p}{)}
         \PY{n}{ax1}\PY{o}{.}\PY{n}{set\PYZus{}xticklabels}\PY{p}{(}\PY{n}{labels}\PY{o}{=}\PY{n}{coeff\PYZus{}df}\PY{o}{.}\PY{n}{index}\PY{p}{,} \PY{n}{size}\PY{o}{=}\PY{l+m+mi}{6}\PY{p}{,} \PY{n}{rotation}\PY{o}{=}\PY{l+m+mi}{90}\PY{p}{)}
         \PY{n}{ax1}\PY{o}{.}\PY{n}{set\PYZus{}ylabel}\PY{p}{(}\PY{l+s+s1}{\PYZsq{}}\PY{l+s+s1}{Importance}\PY{l+s+s1}{\PYZsq{}}\PY{p}{)}
\end{Verbatim}


\begin{Verbatim}[commandchars=\\\{\}]
{\color{outcolor}Out[{\color{outcolor}34}]:} <matplotlib.text.Text at 0x7f3d38df20f0>
\end{Verbatim}
            
    \begin{center}
    \adjustimage{max size={0.9\linewidth}{0.9\paperheight}}{output_15_1.png}
    \end{center}
    { \hspace*{\fill} \\}
    
    \begin{Verbatim}[commandchars=\\\{\}]
{\color{incolor}In [{\color{incolor}36}]:} \PY{n}{fig}\PY{p}{,} \PY{p}{(}\PY{n}{ax1}\PY{p}{,} \PY{n}{ax2}\PY{p}{)} \PY{o}{=} \PY{n}{plt}\PY{o}{.}\PY{n}{subplots}\PY{p}{(}\PY{l+m+mi}{2}\PY{p}{,}\PY{l+m+mi}{1}\PY{p}{,} \PY{n}{figsize}\PY{o}{=}\PY{p}{(}\PY{l+m+mi}{16}\PY{p}{,}\PY{l+m+mi}{16}\PY{p}{)}\PY{p}{)}
         
         \PY{n}{sns}\PY{o}{.}\PY{n}{barplot}\PY{p}{(}\PY{n}{x}\PY{o}{=}\PY{n}{coeff\PYZus{}df}\PY{o}{.}\PY{n}{index}\PY{p}{[}\PY{p}{:}\PY{l+m+mi}{10}\PY{p}{]}\PY{p}{,} \PY{n}{y}\PY{o}{=}\PY{n}{coeff\PYZus{}df}\PY{p}{[}\PY{l+s+s1}{\PYZsq{}}\PY{l+s+s1}{Feature\PYZus{}Import}\PY{l+s+s1}{\PYZsq{}}\PY{p}{]}\PY{o}{.}\PY{n}{head}\PY{p}{(}\PY{l+m+mi}{10}\PY{p}{)}\PY{p}{,} \PY{n}{ax}\PY{o}{=}\PY{n}{ax1}\PY{p}{)}
         \PY{n}{ax1}\PY{o}{.}\PY{n}{set\PYZus{}title}\PY{p}{(}\PY{l+s+s1}{\PYZsq{}}\PY{l+s+s1}{Главные положительные характеристики}\PY{l+s+s1}{\PYZsq{}}\PY{p}{)}
         \PY{n}{ax1}\PY{o}{.}\PY{n}{set\PYZus{}ylabel}\PY{p}{(}\PY{l+s+s1}{\PYZsq{}}\PY{l+s+s1}{Importance}\PY{l+s+s1}{\PYZsq{}}\PY{p}{)}
         \PY{n}{ax1}\PY{o}{.}\PY{n}{set\PYZus{}xticklabels}\PY{p}{(}\PY{n}{labels}\PY{o}{=}\PY{n}{coeff\PYZus{}df}\PY{o}{.}\PY{n}{index}\PY{p}{[}\PY{p}{:}\PY{l+m+mi}{10}\PY{p}{]}\PY{p}{,} \PY{n}{fontsize}\PY{o}{=}\PY{l+m+mi}{8}\PY{p}{,} \PY{n}{rotation}\PY{o}{=}\PY{l+m+mi}{20}\PY{p}{)}
         
         
         \PY{n}{sns}\PY{o}{.}\PY{n}{barplot}\PY{p}{(}\PY{n}{x}\PY{o}{=}\PY{n}{coeff\PYZus{}df}\PY{o}{.}\PY{n}{index}\PY{p}{[}\PY{o}{\PYZhy{}}\PY{l+m+mi}{10}\PY{p}{:}\PY{p}{]}\PY{p}{,} \PY{n}{y}\PY{o}{=}\PY{n}{coeff\PYZus{}df}\PY{p}{[}\PY{l+s+s1}{\PYZsq{}}\PY{l+s+s1}{Feature\PYZus{}Import}\PY{l+s+s1}{\PYZsq{}}\PY{p}{]}\PY{o}{.}\PY{n}{tail}\PY{p}{(}\PY{l+m+mi}{10}\PY{p}{)}\PY{p}{,} \PY{n}{ax}\PY{o}{=}\PY{n}{ax2}\PY{p}{,} \PY{n}{palette}\PY{o}{=}\PY{l+s+s1}{\PYZsq{}}\PY{l+s+s1}{hls}\PY{l+s+s1}{\PYZsq{}}\PY{p}{)}
         \PY{n}{ax2}\PY{o}{.}\PY{n}{set\PYZus{}title}\PY{p}{(}\PY{l+s+s1}{\PYZsq{}}\PY{l+s+s1}{Главные отрицательные характеристики}\PY{l+s+s1}{\PYZsq{}}\PY{p}{)}
         \PY{n}{ax2}\PY{o}{.}\PY{n}{set\PYZus{}ylabel}\PY{p}{(}\PY{l+s+s1}{\PYZsq{}}\PY{l+s+s1}{Importance}\PY{l+s+s1}{\PYZsq{}}\PY{p}{)}
         \PY{n}{ax2}\PY{o}{.}\PY{n}{set\PYZus{}xticklabels}\PY{p}{(}\PY{n}{labels}\PY{o}{=}\PY{n}{coeff\PYZus{}df}\PY{o}{.}\PY{n}{index}\PY{p}{[}\PY{o}{\PYZhy{}}\PY{l+m+mi}{10}\PY{p}{:}\PY{p}{]}\PY{p}{,} \PY{n}{fontsize}\PY{o}{=}\PY{l+m+mi}{8}\PY{p}{,} \PY{n}{rotation}\PY{o}{=}\PY{l+m+mi}{20}\PY{p}{)}
\end{Verbatim}


\begin{Verbatim}[commandchars=\\\{\}]
{\color{outcolor}Out[{\color{outcolor}36}]:} [<matplotlib.text.Text at 0x7f3d38bdd438>,
          <matplotlib.text.Text at 0x7f3d386f63c8>,
          <matplotlib.text.Text at 0x7f3d3873e9e8>,
          <matplotlib.text.Text at 0x7f3d38731518>,
          <matplotlib.text.Text at 0x7f3d3870d048>,
          <matplotlib.text.Text at 0x7f3d3870db38>,
          <matplotlib.text.Text at 0x7f3d38726668>,
          <matplotlib.text.Text at 0x7f3d38729198>,
          <matplotlib.text.Text at 0x7f3d38729c88>,
          <matplotlib.text.Text at 0x7f3d3871e7b8>]
\end{Verbatim}
            
    \begin{center}
    \adjustimage{max size={0.9\linewidth}{0.9\paperheight}}{output_16_1.png}
    \end{center}
    { \hspace*{\fill} \\}
 
\subsection{Вывод}

Из представленных данных можно сделать некоторые выводы. 

Городская молодежь обычно тратит больше денег, чем деревенские жители, что легко 
объясняется большим количеством мест, где можно потратить деньги. 

Так же можно выделить самые затраные привычки (например, танцы или предпочтение тратить 
деньги на внешний вид).
Нельзя не отметить, что пристрастие к  чтению, любовь к языкам, географии, математике, химии и другие "интеллектуальные" занятия присущи бережливым людям.
    % Add a bibliography block to the postdoc
    
    
    \end{document}
