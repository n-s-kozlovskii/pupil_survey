
% Default to the notebook output style

    


% Inherit from the specified cell style.




    
\documentclass[11pt]{article}

    
    
    \usepackage[T1]{fontenc}
    % Nicer default font (+ math font) than Computer Modern for most use cases
    \usepackage{mathpazo}

    % Basic figure setup, for now with no caption control since it's done
    % automatically by Pandoc (which extracts ![](path) syntax from Markdown).
    \usepackage{graphicx}
    % We will generate all images so they have a width \maxwidth. This means
    % that they will get their normal width if they fit onto the page, but
    % are scaled down if they would overflow the margins.
    \makeatletter
    \def\maxwidth{\ifdim\Gin@nat@width>\linewidth\linewidth
    \else\Gin@nat@width\fi}
    \makeatother
    \let\Oldincludegraphics\includegraphics
    % Set max figure width to be 80% of text width, for now hardcoded.
    \renewcommand{\includegraphics}[1]{\Oldincludegraphics[width=.8\maxwidth]{#1}}
    % Ensure that by default, figures have no caption (until we provide a
    % proper Figure object with a Caption API and a way to capture that
    % in the conversion process - todo).
    \usepackage{caption}
    \DeclareCaptionLabelFormat{nolabel}{}
    \captionsetup{labelformat=nolabel}

    \usepackage{adjustbox} % Used to constrain images to a maximum size 
    \usepackage{xcolor} % Allow colors to be defined
    \usepackage{enumerate} % Needed for markdown enumerations to work
    \usepackage{geometry} % Used to adjust the document margins
    \usepackage{amsmath} % Equations
    \usepackage{amssymb} % Equations
    \usepackage{textcomp} % defines textquotesingle
    % Hack from http://tex.stackexchange.com/a/47451/13684:
    \AtBeginDocument{%
        \def\PYZsq{\textquotesingle}% Upright quotes in Pygmentized code
    }
    \usepackage{upquote} % Upright quotes for verbatim code
    \usepackage{eurosym} % defines \euro
    \usepackage[mathletters]{ucs} % Extended unicode (utf-8) support
    \usepackage[utf8x]{inputenc} % Allow utf-8 characters in the tex document
    \usepackage{fancyvrb} % verbatim replacement that allows latex
    \usepackage{grffile} % extends the file name processing of package graphics 
                         % to support a larger range 
    % The hyperref package gives us a pdf with properly built
    % internal navigation ('pdf bookmarks' for the table of contents,
    % internal cross-reference links, web links for URLs, etc.)
    \usepackage{hyperref}
    \usepackage{longtable} % longtable support required by pandoc >1.10
    \usepackage{booktabs}  % table support for pandoc > 1.12.2
    \usepackage[inline]{enumitem} % IRkernel/repr support (it uses the enumerate* environment)
    \usepackage[normalem]{ulem} % ulem is needed to support strikethroughs (\sout)
                                % normalem makes italics be italics, not underlines
    
	\usepackage[russian]{babel}
    
    
    % Colors for the hyperref package
    \definecolor{urlcolor}{rgb}{0,.145,.698}
    \definecolor{linkcolor}{rgb}{.71,0.21,0.01}
    \definecolor{citecolor}{rgb}{.12,.54,.11}

    % ANSI colors
    \definecolor{ansi-black}{HTML}{3E424D}
    \definecolor{ansi-black-intense}{HTML}{282C36}
    \definecolor{ansi-red}{HTML}{E75C58}
    \definecolor{ansi-red-intense}{HTML}{B22B31}
    \definecolor{ansi-green}{HTML}{00A250}
    \definecolor{ansi-green-intense}{HTML}{007427}
    \definecolor{ansi-yellow}{HTML}{DDB62B}
    \definecolor{ansi-yellow-intense}{HTML}{B27D12}
    \definecolor{ansi-blue}{HTML}{208FFB}
    \definecolor{ansi-blue-intense}{HTML}{0065CA}
    \definecolor{ansi-magenta}{HTML}{D160C4}
    \definecolor{ansi-magenta-intense}{HTML}{A03196}
    \definecolor{ansi-cyan}{HTML}{60C6C8}
    \definecolor{ansi-cyan-intense}{HTML}{258F8F}
    \definecolor{ansi-white}{HTML}{C5C1B4}
    \definecolor{ansi-white-intense}{HTML}{A1A6B2}

    % commands and environments needed by pandoc snippets
    % extracted from the output of `pandoc -s`
    \providecommand{\tightlist}{%
      \setlength{\itemsep}{0pt}\setlength{\parskip}{0pt}}
    \DefineVerbatimEnvironment{Highlighting}{Verbatim}{commandchars=\\\{\}}
    % Add ',fontsize=\small' for more characters per line
    \newenvironment{Shaded}{}{}
    \newcommand{\KeywordTok}[1]{\textcolor[rgb]{0.00,0.44,0.13}{\textbf{{#1}}}}
    \newcommand{\DataTypeTok}[1]{\textcolor[rgb]{0.56,0.13,0.00}{{#1}}}
    \newcommand{\DecValTok}[1]{\textcolor[rgb]{0.25,0.63,0.44}{{#1}}}
    \newcommand{\BaseNTok}[1]{\textcolor[rgb]{0.25,0.63,0.44}{{#1}}}
    \newcommand{\FloatTok}[1]{\textcolor[rgb]{0.25,0.63,0.44}{{#1}}}
    \newcommand{\CharTok}[1]{\textcolor[rgb]{0.25,0.44,0.63}{{#1}}}
    \newcommand{\StringTok}[1]{\textcolor[rgb]{0.25,0.44,0.63}{{#1}}}
    \newcommand{\CommentTok}[1]{\textcolor[rgb]{0.38,0.63,0.69}{\textit{{#1}}}}
    \newcommand{\OtherTok}[1]{\textcolor[rgb]{0.00,0.44,0.13}{{#1}}}
    \newcommand{\AlertTok}[1]{\textcolor[rgb]{1.00,0.00,0.00}{\textbf{{#1}}}}
    \newcommand{\FunctionTok}[1]{\textcolor[rgb]{0.02,0.16,0.49}{{#1}}}
    \newcommand{\RegionMarkerTok}[1]{{#1}}
    \newcommand{\ErrorTok}[1]{\textcolor[rgb]{1.00,0.00,0.00}{\textbf{{#1}}}}
    \newcommand{\NormalTok}[1]{{#1}}
    
    % Additional commands for more recent versions of Pandoc
    \newcommand{\ConstantTok}[1]{\textcolor[rgb]{0.53,0.00,0.00}{{#1}}}
    \newcommand{\SpecialCharTok}[1]{\textcolor[rgb]{0.25,0.44,0.63}{{#1}}}
    \newcommand{\VerbatimStringTok}[1]{\textcolor[rgb]{0.25,0.44,0.63}{{#1}}}
    \newcommand{\SpecialStringTok}[1]{\textcolor[rgb]{0.73,0.40,0.53}{{#1}}}
    \newcommand{\ImportTok}[1]{{#1}}
    \newcommand{\DocumentationTok}[1]{\textcolor[rgb]{0.73,0.13,0.13}{\textit{{#1}}}}
    \newcommand{\AnnotationTok}[1]{\textcolor[rgb]{0.38,0.63,0.69}{\textbf{\textit{{#1}}}}}
    \newcommand{\CommentVarTok}[1]{\textcolor[rgb]{0.38,0.63,0.69}{\textbf{\textit{{#1}}}}}
    \newcommand{\VariableTok}[1]{\textcolor[rgb]{0.10,0.09,0.49}{{#1}}}
    \newcommand{\ControlFlowTok}[1]{\textcolor[rgb]{0.00,0.44,0.13}{\textbf{{#1}}}}
    \newcommand{\OperatorTok}[1]{\textcolor[rgb]{0.40,0.40,0.40}{{#1}}}
    \newcommand{\BuiltInTok}[1]{{#1}}
    \newcommand{\ExtensionTok}[1]{{#1}}
    \newcommand{\PreprocessorTok}[1]{\textcolor[rgb]{0.74,0.48,0.00}{{#1}}}
    \newcommand{\AttributeTok}[1]{\textcolor[rgb]{0.49,0.56,0.16}{{#1}}}
    \newcommand{\InformationTok}[1]{\textcolor[rgb]{0.38,0.63,0.69}{\textbf{\textit{{#1}}}}}
    \newcommand{\WarningTok}[1]{\textcolor[rgb]{0.38,0.63,0.69}{\textbf{\textit{{#1}}}}}
    
    
    % Define a nice break command that doesn't care if a line doesn't already
    % exist.
    \def\br{\hspace*{\fill} \\* }
    % Math Jax compatability definitions
    \def\gt{>}
    \def\lt{<}
    % Document parameters
    \title{pupil\_survey}
    
    
    

    % Pygments definitions
    
\makeatletter
\def\PY@reset{\let\PY@it=\relax \let\PY@bf=\relax%
    \let\PY@ul=\relax \let\PY@tc=\relax%
    \let\PY@bc=\relax \let\PY@ff=\relax}
\def\PY@tok#1{\csname PY@tok@#1\endcsname}
\def\PY@toks#1+{\ifx\relax#1\empty\else%
    \PY@tok{#1}\expandafter\PY@toks\fi}
\def\PY@do#1{\PY@bc{\PY@tc{\PY@ul{%
    \PY@it{\PY@bf{\PY@ff{#1}}}}}}}
\def\PY#1#2{\PY@reset\PY@toks#1+\relax+\PY@do{#2}}

\expandafter\def\csname PY@tok@gs\endcsname{\let\PY@bf=\textbf}
\expandafter\def\csname PY@tok@kn\endcsname{\let\PY@bf=\textbf\def\PY@tc##1{\textcolor[rgb]{0.00,0.50,0.00}{##1}}}
\expandafter\def\csname PY@tok@err\endcsname{\def\PY@bc##1{\setlength{\fboxsep}{0pt}\fcolorbox[rgb]{1.00,0.00,0.00}{1,1,1}{\strut ##1}}}
\expandafter\def\csname PY@tok@kr\endcsname{\let\PY@bf=\textbf\def\PY@tc##1{\textcolor[rgb]{0.00,0.50,0.00}{##1}}}
\expandafter\def\csname PY@tok@cs\endcsname{\let\PY@it=\textit\def\PY@tc##1{\textcolor[rgb]{0.25,0.50,0.50}{##1}}}
\expandafter\def\csname PY@tok@gt\endcsname{\def\PY@tc##1{\textcolor[rgb]{0.00,0.27,0.87}{##1}}}
\expandafter\def\csname PY@tok@kd\endcsname{\let\PY@bf=\textbf\def\PY@tc##1{\textcolor[rgb]{0.00,0.50,0.00}{##1}}}
\expandafter\def\csname PY@tok@m\endcsname{\def\PY@tc##1{\textcolor[rgb]{0.40,0.40,0.40}{##1}}}
\expandafter\def\csname PY@tok@nb\endcsname{\def\PY@tc##1{\textcolor[rgb]{0.00,0.50,0.00}{##1}}}
\expandafter\def\csname PY@tok@k\endcsname{\let\PY@bf=\textbf\def\PY@tc##1{\textcolor[rgb]{0.00,0.50,0.00}{##1}}}
\expandafter\def\csname PY@tok@no\endcsname{\def\PY@tc##1{\textcolor[rgb]{0.53,0.00,0.00}{##1}}}
\expandafter\def\csname PY@tok@kt\endcsname{\def\PY@tc##1{\textcolor[rgb]{0.69,0.00,0.25}{##1}}}
\expandafter\def\csname PY@tok@ch\endcsname{\let\PY@it=\textit\def\PY@tc##1{\textcolor[rgb]{0.25,0.50,0.50}{##1}}}
\expandafter\def\csname PY@tok@gh\endcsname{\let\PY@bf=\textbf\def\PY@tc##1{\textcolor[rgb]{0.00,0.00,0.50}{##1}}}
\expandafter\def\csname PY@tok@s\endcsname{\def\PY@tc##1{\textcolor[rgb]{0.73,0.13,0.13}{##1}}}
\expandafter\def\csname PY@tok@sx\endcsname{\def\PY@tc##1{\textcolor[rgb]{0.00,0.50,0.00}{##1}}}
\expandafter\def\csname PY@tok@sb\endcsname{\def\PY@tc##1{\textcolor[rgb]{0.73,0.13,0.13}{##1}}}
\expandafter\def\csname PY@tok@nn\endcsname{\let\PY@bf=\textbf\def\PY@tc##1{\textcolor[rgb]{0.00,0.00,1.00}{##1}}}
\expandafter\def\csname PY@tok@gu\endcsname{\let\PY@bf=\textbf\def\PY@tc##1{\textcolor[rgb]{0.50,0.00,0.50}{##1}}}
\expandafter\def\csname PY@tok@mb\endcsname{\def\PY@tc##1{\textcolor[rgb]{0.40,0.40,0.40}{##1}}}
\expandafter\def\csname PY@tok@bp\endcsname{\def\PY@tc##1{\textcolor[rgb]{0.00,0.50,0.00}{##1}}}
\expandafter\def\csname PY@tok@c\endcsname{\let\PY@it=\textit\def\PY@tc##1{\textcolor[rgb]{0.25,0.50,0.50}{##1}}}
\expandafter\def\csname PY@tok@dl\endcsname{\def\PY@tc##1{\textcolor[rgb]{0.73,0.13,0.13}{##1}}}
\expandafter\def\csname PY@tok@vm\endcsname{\def\PY@tc##1{\textcolor[rgb]{0.10,0.09,0.49}{##1}}}
\expandafter\def\csname PY@tok@sh\endcsname{\def\PY@tc##1{\textcolor[rgb]{0.73,0.13,0.13}{##1}}}
\expandafter\def\csname PY@tok@sd\endcsname{\let\PY@it=\textit\def\PY@tc##1{\textcolor[rgb]{0.73,0.13,0.13}{##1}}}
\expandafter\def\csname PY@tok@ss\endcsname{\def\PY@tc##1{\textcolor[rgb]{0.10,0.09,0.49}{##1}}}
\expandafter\def\csname PY@tok@ne\endcsname{\let\PY@bf=\textbf\def\PY@tc##1{\textcolor[rgb]{0.82,0.25,0.23}{##1}}}
\expandafter\def\csname PY@tok@sa\endcsname{\def\PY@tc##1{\textcolor[rgb]{0.73,0.13,0.13}{##1}}}
\expandafter\def\csname PY@tok@sc\endcsname{\def\PY@tc##1{\textcolor[rgb]{0.73,0.13,0.13}{##1}}}
\expandafter\def\csname PY@tok@gd\endcsname{\def\PY@tc##1{\textcolor[rgb]{0.63,0.00,0.00}{##1}}}
\expandafter\def\csname PY@tok@kp\endcsname{\def\PY@tc##1{\textcolor[rgb]{0.00,0.50,0.00}{##1}}}
\expandafter\def\csname PY@tok@gp\endcsname{\let\PY@bf=\textbf\def\PY@tc##1{\textcolor[rgb]{0.00,0.00,0.50}{##1}}}
\expandafter\def\csname PY@tok@s1\endcsname{\def\PY@tc##1{\textcolor[rgb]{0.73,0.13,0.13}{##1}}}
\expandafter\def\csname PY@tok@s2\endcsname{\def\PY@tc##1{\textcolor[rgb]{0.73,0.13,0.13}{##1}}}
\expandafter\def\csname PY@tok@cm\endcsname{\let\PY@it=\textit\def\PY@tc##1{\textcolor[rgb]{0.25,0.50,0.50}{##1}}}
\expandafter\def\csname PY@tok@vc\endcsname{\def\PY@tc##1{\textcolor[rgb]{0.10,0.09,0.49}{##1}}}
\expandafter\def\csname PY@tok@ni\endcsname{\let\PY@bf=\textbf\def\PY@tc##1{\textcolor[rgb]{0.60,0.60,0.60}{##1}}}
\expandafter\def\csname PY@tok@w\endcsname{\def\PY@tc##1{\textcolor[rgb]{0.73,0.73,0.73}{##1}}}
\expandafter\def\csname PY@tok@se\endcsname{\let\PY@bf=\textbf\def\PY@tc##1{\textcolor[rgb]{0.73,0.40,0.13}{##1}}}
\expandafter\def\csname PY@tok@nd\endcsname{\def\PY@tc##1{\textcolor[rgb]{0.67,0.13,1.00}{##1}}}
\expandafter\def\csname PY@tok@nc\endcsname{\let\PY@bf=\textbf\def\PY@tc##1{\textcolor[rgb]{0.00,0.00,1.00}{##1}}}
\expandafter\def\csname PY@tok@kc\endcsname{\let\PY@bf=\textbf\def\PY@tc##1{\textcolor[rgb]{0.00,0.50,0.00}{##1}}}
\expandafter\def\csname PY@tok@ge\endcsname{\let\PY@it=\textit}
\expandafter\def\csname PY@tok@vg\endcsname{\def\PY@tc##1{\textcolor[rgb]{0.10,0.09,0.49}{##1}}}
\expandafter\def\csname PY@tok@nt\endcsname{\let\PY@bf=\textbf\def\PY@tc##1{\textcolor[rgb]{0.00,0.50,0.00}{##1}}}
\expandafter\def\csname PY@tok@si\endcsname{\let\PY@bf=\textbf\def\PY@tc##1{\textcolor[rgb]{0.73,0.40,0.53}{##1}}}
\expandafter\def\csname PY@tok@mi\endcsname{\def\PY@tc##1{\textcolor[rgb]{0.40,0.40,0.40}{##1}}}
\expandafter\def\csname PY@tok@c1\endcsname{\let\PY@it=\textit\def\PY@tc##1{\textcolor[rgb]{0.25,0.50,0.50}{##1}}}
\expandafter\def\csname PY@tok@cpf\endcsname{\let\PY@it=\textit\def\PY@tc##1{\textcolor[rgb]{0.25,0.50,0.50}{##1}}}
\expandafter\def\csname PY@tok@o\endcsname{\def\PY@tc##1{\textcolor[rgb]{0.40,0.40,0.40}{##1}}}
\expandafter\def\csname PY@tok@fm\endcsname{\def\PY@tc##1{\textcolor[rgb]{0.00,0.00,1.00}{##1}}}
\expandafter\def\csname PY@tok@nv\endcsname{\def\PY@tc##1{\textcolor[rgb]{0.10,0.09,0.49}{##1}}}
\expandafter\def\csname PY@tok@nf\endcsname{\def\PY@tc##1{\textcolor[rgb]{0.00,0.00,1.00}{##1}}}
\expandafter\def\csname PY@tok@mf\endcsname{\def\PY@tc##1{\textcolor[rgb]{0.40,0.40,0.40}{##1}}}
\expandafter\def\csname PY@tok@gi\endcsname{\def\PY@tc##1{\textcolor[rgb]{0.00,0.63,0.00}{##1}}}
\expandafter\def\csname PY@tok@mo\endcsname{\def\PY@tc##1{\textcolor[rgb]{0.40,0.40,0.40}{##1}}}
\expandafter\def\csname PY@tok@il\endcsname{\def\PY@tc##1{\textcolor[rgb]{0.40,0.40,0.40}{##1}}}
\expandafter\def\csname PY@tok@mh\endcsname{\def\PY@tc##1{\textcolor[rgb]{0.40,0.40,0.40}{##1}}}
\expandafter\def\csname PY@tok@gr\endcsname{\def\PY@tc##1{\textcolor[rgb]{1.00,0.00,0.00}{##1}}}
\expandafter\def\csname PY@tok@nl\endcsname{\def\PY@tc##1{\textcolor[rgb]{0.63,0.63,0.00}{##1}}}
\expandafter\def\csname PY@tok@ow\endcsname{\let\PY@bf=\textbf\def\PY@tc##1{\textcolor[rgb]{0.67,0.13,1.00}{##1}}}
\expandafter\def\csname PY@tok@go\endcsname{\def\PY@tc##1{\textcolor[rgb]{0.53,0.53,0.53}{##1}}}
\expandafter\def\csname PY@tok@sr\endcsname{\def\PY@tc##1{\textcolor[rgb]{0.73,0.40,0.53}{##1}}}
\expandafter\def\csname PY@tok@cp\endcsname{\def\PY@tc##1{\textcolor[rgb]{0.74,0.48,0.00}{##1}}}
\expandafter\def\csname PY@tok@na\endcsname{\def\PY@tc##1{\textcolor[rgb]{0.49,0.56,0.16}{##1}}}
\expandafter\def\csname PY@tok@vi\endcsname{\def\PY@tc##1{\textcolor[rgb]{0.10,0.09,0.49}{##1}}}

\def\PYZbs{\char`\\}
\def\PYZus{\char`\_}
\def\PYZob{\char`\{}
\def\PYZcb{\char`\}}
\def\PYZca{\char`\^}
\def\PYZam{\char`\&}
\def\PYZlt{\char`\<}
\def\PYZgt{\char`\>}
\def\PYZsh{\char`\#}
\def\PYZpc{\char`\%}
\def\PYZdl{\char`\$}
\def\PYZhy{\char`\-}
\def\PYZsq{\char`\'}
\def\PYZdq{\char`\"}
\def\PYZti{\char`\~}
% for compatibility with earlier versions
\def\PYZat{@}
\def\PYZlb{[}
\def\PYZrb{]}
\makeatother


    % Exact colors from NB
    \definecolor{incolor}{rgb}{0.0, 0.0, 0.5}
    \definecolor{outcolor}{rgb}{0.545, 0.0, 0.0}



    
    % Prevent overflowing lines due to hard-to-break entities
    \sloppy 
    % Setup hyperref package
    \hypersetup{
      breaklinks=true,  % so long urls are correctly broken across lines
      colorlinks=true,
      urlcolor=urlcolor,
      linkcolor=linkcolor,
      citecolor=citecolor,
      }
    % Slightly bigger margins than the latex defaults
    
    \geometry{verbose,tmargin=1in,bmargin=1in,lmargin=1in,rmargin=1in}
    
    

    \begin{document}
    
    
    \title{Анализ опроса словацкой молодежи 2013}
    \author
{Никита Козловский\\
\\
\normalsize{ВГУ, ПММ, 3 группа}\\}
\date{}
\maketitle

\section{Введение}
    Рассмотрим данные, представленные на сайте \url{https://www.kaggle.com/miroslavsabo/young-people-survey}
	Опрос был проведен факультетом факультет социальных и экономических наук Университета имени Коменского в Братиславе в 2013 году. Было опрошено 1010 человек по 150 пунктам: 
	\begin{itemize}
\item Предпочтения в музыке (19 пунктов)
\item Предпочтения в фильмах (12 пунктов)
\item Хобби и интересы (32 пунктов)
\item Фобии (10 пунктов)
\item Здравоохранение (3 пунктов)
\item Черты характера, взгляды на жизнь и мнения (57 пунктов)
\item На что тратите деньги (7 пунктов)
\item Демографические данные (10 пунктов)
\end{itemize}

Попробуем найти ответы на следующие вопросы:
	\begin{itemize}
\item Можно ли разделить студентов на группы по предпочтениям в музыке? (описательные, факторный )
\item Что определяет бережливого человека ? (регрессия) 
\item Анализ страхов по группам 
\end{itemize}
    
\pagebreak
\section{Анкета}
\subsection{МУЗЫКАЛЬНЫЕ ПРЕДПОЧТЕНИЯ}
\begin{enumerate}
\item Мне нравится слушать музыку: Категорически не согласен 1-2-3-4-5 Полностью согласен (целое)
\item Я предпочитаю: Медленную музыку 1-2-3-4-5 Быструю музыку (целое)
\item Dance, Disco, Funk: Не нравится вообще1-2-3-4-5 Нравится очень (целое)
\item Народная музыка: Не нравится вообще 1-2-3-4-5 люблю (целое)
\item Кантри: Не нравится вообще 1-2-3-4-5 Нравится очень (целое)
\item Классика: Не нравится вообще 1-2-3-4-5 Нравится очень (целое)
\item Мюзиклы: Не нравится вообще 1-2-3-4-5 Нравится очень (целое)
\item Поп: Не нравится вообще 1-2-3-4-5 Нравится очень (целое)
\item Рок: Не нравится вообще 1-2-3-4-5 Нравится очень (целое)
\item Металл, хард-рок: Не нравится вообще 1-2-3-4-5 Нравится очень (целое)
\item Панк: Не нравится вообще  1-2-3-4-5 Нравится очень (целое)
\item Хип-хоп, Рэп: Не нравится вообще 1-2-3-4-5 Нравится очень (целое)
\item Reggae, Ska: Не нравится вообще 1-2-3-4-5 Нравится очень (целое)
\item Swing, Jazz: не нравится вообще 1-2-3-4-5 Нравится очень (целое)
\item Рок-н-ролл: не нравится вообще 1-2-3-4-5 Нравится очень (целое)
\item Альтернативная музыка: не нравится вообще 1-2-3-4-5 Нравится очень (целое)
\item Латино: не нравится вообще 1-2-3-4-5 Нравится очень (целое)
\item Техно, Транс: не нравится вообще  1-2-3-4-5 Нравится очень (целое)
\item Опера: не нравится вообще  1-2-3-4-5 Нравится очень (целое)
\end{enumerate}
\subsection{ФИЛЬМЫ ПРЕДПОЧТЕНИЯ}
\begin{enumerate}
\item Мне очень нравится смотреть фильмы: Сильно не согласен 1-2-3-4-5 Полностью согласен (целое)
\item Фильмы ужасов: Не нравится вообще 1-2-3-4-5 Нравится очень (целое)
\item Триллеры: Не нравится вообще 1-2-3-4-5 Нравится очень (целое)
\item Комедии: не нравится вообще 1-2-3-4-5 Нравится очень (целое)
\item Романтические фильмы: Не нравится вообще 1-2-3-4-5 Нравится очень (целое)
\item Научно-фантастические фильмы: Не нравится вообще 1-2-3-4-5 Нравится очень (целое)
\item Военные фильмы: Не нравится вообще 1-2-3-4-5 Нравится очень (целое)
\item Сказки: Не нравится вообще 1-2-3-4-5 Нравится очень (целое)
\item Мультфильмы: Не нравится вообще 1-2-3-4-5 Наслаждайтесь очень (целое)
\item Документальные фильмы: Не нравится вообще 1-2-3-4-5 Нравится очень (целое)
\item Западные фильмы: Не нравится вообще 1-2-3-4-5 Нравится очень (целое)
\item Экшн фильмы: Не нравится вообще 1-2-3-4-5 Нравится очень (целое)
\end{enumerate}
\subsection{ХОББИ ИНТЕРЕСЫ}
\begin{enumerate}
\item История: Не интересует 1-2-3-4-5 Очень интересно (целое)
\item Психология: не интересует 1-2-3-4-5 Очень интересно (целое)
\item Политика: Не интересует 1-2-3-4-5 Очень интересно (целое)
\item Математика: не интересует 1-2-3-4-5 Очень интересно (целое)
\item Физика: не интересует 1-2-3-4-5 Очень интересно (целое)
\item Интернет: не интересует 1-2-3-4-5 Очень интересно (целое)
\item Программное обеспечение для ПК, Оборудование: Не интересует 1-2-3-4-5 Очень интересно (целое)
\item Экономика, Менеджмент: Не интересует 1-2-3-4-5 Очень интересно (целое)
\item Биология: не интересует 1-2-3-4-5 Очень интересно (целое)
\item Химия: не интересует 1-2-3-4-5 Очень интересно (целое)
\item Чтение стихов: Не интересует 1-2-3-4-5 Очень интересно (целое)
\item География: Не интересует 1-2-3-4-5 Очень интересно (целое)
\item Иностранные языки: не интересуются 1-2-3-4-5 Очень интересно (целое)
\item Медицина: не интересует 1-2-3-4-5 Очень интересно (целое)
\item Закон: не интересует 1-2-3-4-5 Очень интересно (целое)
\item Автомобили: Не интересует 1-2-3-4-5 Очень интересно (целое)
\item Искусство: Не интересует 1-2-3-4-5 Очень интересно (целое)
\item Религия: Не интересует 1-2-3-4-5 Очень интересно (целое)
\item Мероприятия на свежем воздухе: не интересует 1-2-3-4-5 Очень интересно (целое)
\item Танцы: не интересуются 1-2-3-4-5 Очень интересно (целое)
\item Игра на музыкальных инструментах: не интересует 1-2-3-4-5 Очень интересно (целое)
\item Поэзия: Не интересно 1-2-3-4-5 Очень интересно (целое)
\item Занятия спортом: Не интересуюсь 1-2-3-4-5 Очень интересно (целое)
\item Спорт на конкурентном уровне: не интересует 1-2-3-4-5 Очень интересно (целое)
\item Садоводство: Не интересует 1-2-3-4-5 Очень интересно (целое)
\item Знаменитый образ жизни: не интересует 1-2-3-4-5 Очень интересно (целое)
\item Покупки: Не интересует 1-2-3-4-5 Очень интересно (целое)
\item Наука и техника: не интересует 1-2-3-4-5 Очень интересно (целое)
\item Театр: не интересует 1-2-3-4-5 Очень интересно (целое)
\item Общение: Не интересует 1-2-3-4-5 Очень интересно (целое)
\item Адреналин спорт: не интересует 1-2-3-4-5 Очень интересно (целое)
\item Домашние животные: Не интересуются 1-2-3-4-5 Очень интересно (целое)
\end{enumerate}
\subsection{ФОБИИ}
\begin{enumerate}
\item Летающий: совсем не боится 1-2-3-4-5 Очень боится (целое)
\item Гром, молния: совсем не боюсь 1-2-3-4-5 Очень боится (целое)
\item Тьма: не боится совсем 1-2-3-4-5 Очень боится (целое)
\item Высота: совсем не боится 1-2-3-4-5 Очень боится (целое)
\item Пауки: совсем не боюсь 1-2-3-4-5 Очень боится (целое)
\item Змеи: совсем не боюсь 1-2-3-4-5 Очень боится (целое)
\item Крысы, мыши: совсем не боятся 1-2-3-4-5 Очень боится (целое)
\end{enumerate}
\subsection{ЗДОРОВЬЕ}
\begin{enumerate}
\item Отношение к курению: Никогда не курил - Пробовал курить - Бывший курильщик - Текущий курильщик (Номинальная)
\item Пью: Никогда - Пью в компаниях - Пью много (Номинальная)
\item Я живу очень здоровым образом жизни: сильно не согласен 1-2-3-4-5 Полностью согласен (целое)
\end{enumerate}
\subsection{ЧЕРТЫ ХАРАКТЕРА, ВЗГЛЯДЫ НА ЖИЗНЬ}
\begin{enumerate}
\item Я обращаю внимание на то, что происходит вокруг: сильно не согласен 1-2-3-4-5. Полностью согласен (целое)
\item Я стараюсь выполнять задания как можно скорее и не оставлять их до последней минуты .: Сильно не согласен 1-2-3-4-5 Полностью согласен (целое)
\item Я всегда составляю список, поэтому ничего не забываю: сильно не согласен 1-2-3-4-5 Полностью согласен (целое)
\item Я часто учусь или работаю даже в свободное время .: Сильно не согласен 1-2-3-4-5 Полностью согласен (целое)
\item Я смотрю на вещи с разных точек зрения, прежде чем идти вперед. Полностью не согласен 1-2-3-4-5. Полностью согласен (целое)
\item Я считаю, что плохие люди пострадают в один прекрасный день, и хорошие люди будут вознаграждены. Полностью не согласен 1-2-3-4-5. Полностью согласен (целое)
\item Я надёжен на работе и всегда выполняю все заданные мне задачи: Полностью не согласен 1-2-3-4-5 Полностью согласен (целое)
\item Я всегда соблюдаю свои обещания: сильно не согласен 1-2-3-4-5 Полностью согласен (целое)
\item Я могу быстро попасть на кого-то, а затем полностью потерять интерес. Полностью не согласен 1-2-3-4-5. Полностью согласен (целое)
\item Я бы предпочел иметь много друзей, чем много денег. Полностью не согласен 1-2-3-4-5. Полностью согласен (целое)
\item Я всегда стараюсь быть самым смешным: сильно не согласен 1-2-3-4-5 Полностью согласен (целое)
\item Иногда я могу столкнуться с двумя лицами: Полностью не согласен 1-2-3-4-5 Полностью согласен (целое)
\item Я повредил вещи в прошлом, когда сердился .: Сильно не согласен 1-2-3-4-5 Полностью согласен (целое)
\item Я не тороплюсь, чтобы принимать решения: сильно не согласен 1-2-3-4-5 Полностью согласен (целое)
\item Я всегда стараюсь голосовать на выборах: сильно не согласен 1-2-3-4-5 Полностью согласен (целое)
\item Я часто думаю и сожалею о принимаемых мной решениях: Полностью не согласен 1-2-3-4-5 Полностью согласен (целое)
\item Я могу сказать, слушают ли меня люди или нет, когда я говорю с ними: Полностью не согласен 1-2-3-4-5 Полностью согласен (целое)
\item Я ипохондрик. Полностью не согласен 1-2-3-4-5. Полностью согласен (целое)
\item Я - эмфатичный человек. Сильно не согласен 1-2-3-4-5 Полностью согласен (целое)
\item Я ем, потому что должен. Я не наслаждаюсь едой и еду так быстро, как могу: Категорически не согласен 1-2-3-4-5 Полностью согласен (целое)
\item Я стараюсь отдать столько, сколько я могу, другим людям на Рождество: сильно не согласен 1-2-3-4-5 Полностью согласен (целое)
\item Мне не нравится встречаться с животными: Полностью не согласен 1-2-3-4-5 Полностью согласен (целое)
\item Я ухаживаю за вещами, которые я заимствовал у других: сильно не согласен 1-2-3-4-5 Полностью согласен (целое)
\item Я чувствую себя одиноким в жизни: Сильно не согласен 1-2-3-4-5 Полностью согласен (целое)
\item Раньше я учился в школе: Сильно не согласен 1-2-3-4-5 Полностью согласен (целое)
\item Я беспокоюсь о своем здоровье: сильно не согласен 1-2-3-4-5 Полностью согласен (целое)
\item Хотел бы я изменить прошлое из-за того, что я сделал: Полностью не согласен 1-2-3-4-5 Сильно соглашаюсь (целое)
\item Я верю в Бога: Полностью не согласен 1-2-3-4-5 Полностью согласен (целое)
\item У меня всегда хорошие мечты: сильно не согласен 1-2-3-4-5 Полностью согласен (целое)
\item Я всегда отдаю благотворительность. Полностью не согласен 1-2-3-4-5. Полностью согласен (целое)
\item У меня много друзей: сильно не согласен 1-2-3-4-5 Полностью согласен (целое)
\item Сроки: Я часто бываю на ранней стадии. - Я всегда вовремя. - Я часто опаздываю. (Категорично)
\item Вы лжете другим: Никогда. - Только чтобы не причинять кому-либо вреда. Иногда. - Каждый раз мне это подходит. (Номинальная)
\item Я очень терпелив: сильно не согласен 1-2-3-4-5 Полностью согласен (целое)
\item Я могу быстро адаптироваться к новой среде .: Сильно не согласен 1-2-3-4-5 Полностью согласен (целое)
\item Мое настроение меняется быстро: сильно не согласен 1-2-3-4-5 Полностью согласен (целое)
\item Я хорошо воспитан, и я ухаживаю за внешностью: Полностью не согласен 1-2-3-4-5 Полностью согласен (целое)
\item Мне нравится встречаться с новыми людьми .: Сильно не согласен 1-2-3-4-5 Полностью согласен (целое)
\item Я всегда позволяю другим людям узнать о моих достижениях: Полностью не согласен 1-2-3-4-5 Полностью согласен (целое)
\item Я думаю, тщательно, прежде чем отвечать на любые важные письма .: Полностью не согласен 1-2-3-4-5 Полностью согласен (целое)
\item Я люблю детей: сильно не согласен 1-2-3-4-5 Полностью согласен (целое)
\item Я не боюсь высказать свое мнение, если я сильно против чего либо: Полностью не согласен 1-2-3-4-5 Полностью согласен (целое)
\item Я могу рассердиться очень легко: сильно не согласен 1-2-3-4-5 Полностью согласен (целое)
\item Я всегда убеждаюсь, что я общаюсь с нужными людьми: сильно не согласен 1-2-3-4-5 Полностью согласен (целое)
\item Я должен быть хорошо подготовлен перед публичным выступлением: Полностью не согласен 1-2-3-4-5 Полностью согласен (целое)
\item Я виноват в том, люди меня не любят. Полностью не согласен 1-2-3-4-5. Полностью согласен (целое)
\item Я плачу, когда чувствую себя, или все идет не так. Абсолютно не согласен 1-2-3-4-5. Полностью согласен (целое)
\item Я на 100\% доволен своей жизнью. Полностью не согласен 1-2-3-4-5. Полностью согласен (целое)
\item Я всегда полна жизни и энергии: Полностью не согласен 1-2-3-4-5 Полностью согласен (целое)
\item Я предпочитаю крупных опасных собак более мелким, более спокойным собакам: Полностью не согласен 1-2-3-4-5 Полностью согласен (целое)
\item Я считаю, что все мои черты характера положительные: сильно не согласны 1-2-3-4-5. Полностью согласен (целое)
\item Если я найду что-то, что не принадлежит мне, я передам его. Полностью не согласен 1-2-3-4-5. Полностью согласен (целое)
\item Мне очень трудно вставать утром: сильно не согласен 1-2-3-4-5 Полностью согласен (целое)
\item У меня много разных увлечений и интересов: Полностью не согласен 1-2-3-4-5 Полностью согласен (целое)
\item Я всегда слушаю совет моих родителей: Полностью не согласен 1-2-3-4-5. Полностью согласен (целое)
\item Мне нравится участвовать в опросах: сильно не согласен 1-2-3-4-5 Полностью согласен (целое)
\item Сколько времени вы проводите в Интернете ?: Не провожу - Менее часа в день - Несколько часов в день - Большая часть дня (Номинальная)
\end{enumerate}
\subsection{НА ЧТО ВЫ ТРАТИТЕ ДЕНЬГИ}
\begin{enumerate}
\item Я могу сэкономить все, что смогу. Полностью не согласен 1-2-3-4-5. Полностью согласен (целое)
\item Мне нравится ходить в крупные торговые центры .: Сильно не согласен 1-2-3-4-5 Полностью согласен (целое)
\item Я предпочитаю фирменную одежду: сильно не согласен 1-2-3-4-5 Полностью согласен (целое)
\item Я трачу много денег на вечеринки и общение: Сильно не согласен 1-2-3-4-5 Полностью согласен (целое)
\item Я трачу много денег на внешний вид: сильно не согласен 1-2-3-4-5 Полностью согласен (целое)
\item Я трачу много денег на гаджеты: Полностью не согласен 1-2-3-4-5 Полностью согласен (целое)
\item Я буду с удовольствием  платить больше денег за хорошее, качественное или здоровое питание: Полностью не согласен 1-2-3-4-5 Полностью согласен (целое)
\end{enumerate}
\subsection{ДАННЫЕ О СЕБЕ}
\begin{enumerate}
\item Возраст: (целое)
\item Высота: (целое)
\item Вес: (целое)
\item Сколько у вас братьев и сестер ?: (целое)
\item Пол: Женский - Мужской (Номинальная)
\item Я: Левая рука - Правая (Номинальная)
\item Высшее образование: В настоящее время ученик начальной школы - Начальная школа - Средняя школа - Колледж / Степень бакалавра (Номинальная)
\item Я единственный ребенок: Нет - Да (Номинальная)
\item Я провел большую часть своего детства в: Городе -  деревне (Номинальная)
\item Я прожил большую часть своего детства в: доме  - многоквартирном доме (Номинальная)
\end{enumerate}
\pagebreak

\section{Описательные статистики}
Из-за большого количества признаков, рассмотрим лишь подможество из множества ответов, а именно рассмотрим
предпочтения в музыке. Для каждой характериски расчитаем 

\begin{itemize}
\item mean – средняя величина из исходных значений $\bar{x} = \frac{x_1+x_2+\cdots +x_n}{n}$;
\item std  –  стандартное отклонение, мера того, насколько широко разбросаны точки данных относительно их средних $s = \sqrt{\frac{1}{N-1} \sum_{i=1}^N (x_i - \overline{x})^2}$
\item min, max -- минимальное и максимальное значение
\item 25\% , 50\% , 75\% квантили соответствующих уровней. Значение, которое заданная случайная величина не превышает с фиксированной вероятностью. Верхняя квантиль включает 25\% наибольших чисел в наборе, нижняя, соответственно, 25\% наименьших чисел в наборе, 50\% соответсвует медиане выборки. 
\item skewness – коэффициент асимметрии, который показывает, насколько симметрично распределение данной случайной величины. Если коэффициент асимметрии положителен, то отклонение происходит в сторону положительных значений, в ином случае – отрицательных.
\item kurtosis – эксцесс – мера крутости кривой распределения. Кривая распределения может быть островершинной, плосковершинной, средне вершинной. Эти четыре момента составляют набор особенностей распределения при анализе данных. Для нормального распределения А=0, Е=0. Положительный эксцесс обозначает относительно остроконечное распределение. Отрицательный эксцесс обозначает относительно сглаженное распределение.
\end{itemize}
  
    \begin{Verbatim}[commandchars=\\\{\}]
{\color{incolor}In [{\color{incolor}11}]:} \PY{k+kn}{import} \PY{n+nn}{pandas} \PY{k}{as} \PY{n+nn}{pd}
         \PY{k+kn}{import} \PY{n+nn}{numpy} \PY{k}{as} \PY{n+nn}{np}
\end{Verbatim}


   \begin{Verbatim}[commandchars=\\\{\}]
{\color{incolor}In [{\color{incolor}4}]:} \PY{k+kn}{from} \PY{n+nn}{pandas} \PY{k}{import} \PY{n}{Series}
        \PY{k}{def} \PY{n+nf}{customDescribe}\PY{p}{(}\PY{n}{x}\PY{p}{)}\PY{p}{:}
            \PY{n}{data} \PY{o}{=} \PY{p}{[}\PY{n}{x}\PY{o}{.}\PY{n}{mean}\PY{p}{(}\PY{p}{)}\PY{p}{,} \PY{n}{x}\PY{o}{.}\PY{n}{std}\PY{p}{(}\PY{p}{)}\PY{p}{,} \PY{n}{x}\PY{o}{.}\PY{n}{min}\PY{p}{(}\PY{p}{)}\PY{p}{,} \PY{n}{x}\PY{o}{.}\PY{n}{quantile}\PY{p}{(}\PY{l+m+mf}{0.25}\PY{p}{)}\PY{p}{,} \PY{n}{x}\PY{o}{.}\PY{n}{median}\PY{p}{(}\PY{p}{)}\PY{p}{,} 
                    \PY{n}{x}\PY{o}{.}\PY{n}{quantile}\PY{p}{(}\PY{l+m+mf}{0.75}\PY{p}{)}\PY{p}{,} \PY{n}{x}\PY{o}{.}\PY{n}{max}\PY{p}{(}\PY{p}{)}\PY{p}{,} \PY{n}{x}\PY{o}{.}\PY{n}{skew}\PY{p}{(}\PY{p}{)}\PY{p}{,} \PY{n}{x}\PY{o}{.}\PY{n}{kurtosis}\PY{p}{(}\PY{p}{)}\PY{p}{,} \PY{n}{x}\PY{o}{.}\PY{n}{mode}\PY{p}{(}\PY{p}{)}\PY{o}{.}\PY{n}{max}\PY{p}{(}\PY{p}{)}\PY{p}{,} 
                    \PY{n}{x}\PY{o}{.}\PY{n}{isnull}\PY{p}{(}\PY{p}{)}\PY{o}{.}\PY{n}{sum}\PY{p}{(}\PY{p}{)}\PY{p}{]}
            \PY{n}{names} \PY{o}{=} \PY{p}{[}\PY{l+s+s1}{\PYZsq{}}\PY{l+s+s1}{mean}\PY{l+s+s1}{\PYZsq{}}\PY{p}{,} \PY{l+s+s1}{\PYZsq{}}\PY{l+s+s1}{std}\PY{l+s+s1}{\PYZsq{}}\PY{p}{,} \PY{l+s+s1}{\PYZsq{}}\PY{l+s+s1}{min}\PY{l+s+s1}{\PYZsq{}}\PY{p}{,} \PY{l+s+s1}{\PYZsq{}}\PY{l+s+s1}{25}\PY{l+s+s1}{\PYZpc{}}\PY{l+s+s1}{\PYZsq{}}\PY{p}{,} \PY{l+s+s1}{\PYZsq{}}\PY{l+s+s1}{50}\PY{l+s+s1}{\PYZpc{}}\PY{l+s+s1}{\PYZsq{}}\PY{p}{,} \PY{l+s+s1}{\PYZsq{}}\PY{l+s+s1}{75}\PY{l+s+s1}{\PYZpc{}}\PY{l+s+s1}{\PYZsq{}}\PY{p}{,} \PY{l+s+s1}{\PYZsq{}}\PY{l+s+s1}{max}\PY{l+s+s1}{\PYZsq{}}\PY{p}{,}
             \PY{l+s+s1}{\PYZsq{}}\PY{l+s+s1}{skewness}\PY{l+s+s1}{\PYZsq{}}\PY{p}{,} \PY{l+s+s1}{\PYZsq{}}\PY{l+s+s1}{kurtosis}\PY{l+s+s1}{\PYZsq{}}\PY{p}{,} \PY{l+s+s1}{\PYZsq{}}\PY{l+s+s1}{mode}\PY{l+s+s1}{\PYZsq{}}\PY{p}{,} \PY{l+s+s1}{\PYZsq{}}\PY{l+s+s1}{NAs}\PY{l+s+s1}{\PYZsq{}}\PY{p}{]}
            \PY{k}{return} \PY{n}{Series}\PY{p}{(}\PY{n}{data}\PY{p}{,} \PY{n}{index}\PY{o}{=}\PY{n}{names}\PY{p}{)}
        
        \PY{n}{names} \PY{o}{=} \PY{n}{pd}\PY{o}{.}\PY{n}{read\PYZus{}csv}\PY{p}{(}\PY{l+s+s1}{\PYZsq{}}\PY{l+s+s1}{columns.csv}\PY{l+s+s1}{\PYZsq{}}\PY{p}{)}
        \PY{n}{df} \PY{o}{=} \PY{n}{pd}\PY{o}{.}\PY{n}{read\PYZus{}csv}\PY{p}{(}\PY{l+s+s1}{\PYZsq{}}\PY{l+s+s1}{responses.csv}\PY{l+s+s1}{\PYZsq{}}\PY{p}{)}
        \PY{n}{music} \PY{o}{=} \PY{n}{df}\PY{o}{.}\PY{n}{iloc}\PY{p}{[}\PY{p}{:}\PY{p}{,}\PY{l+m+mi}{2}\PY{p}{:}\PY{l+m+mi}{18}\PY{p}{]}
        \PY{n}{music}\PY{o}{.}\PY{n}{apply}\PY{p}{(}\PY{n}{customDescribe}\PY{p}{)}
\end{Verbatim}


\begin{Verbatim}[commandchars=\\\{\}]
{\color{outcolor}Out[{\color{outcolor}4}]:}              Dance      Folk   Country  Classical music   Musical       Pop  \textbackslash{}
        mean      3.113320  2.288557  2.123383         2.956132  2.761905  3.471698   
        std       1.170568  1.138916  1.076136         1.252570  1.260845  1.161400   
        min       1.000000  1.000000  1.000000         1.000000  1.000000  1.000000   
        25\%       2.000000  1.000000  1.000000         2.000000  2.000000  3.000000   
        50\%       3.000000  2.000000  2.000000         3.000000  3.000000  4.000000   
        75\%       4.000000  3.000000  3.000000         4.000000  4.000000  4.000000   
        max       5.000000  5.000000  5.000000         5.000000  5.000000  5.000000   
        skewness -0.045760  0.694783  0.795798         0.107357  0.219951 -0.383317   
        kurtosis -0.803331 -0.216416 -0.037576        -0.969287 -0.928080 -0.704309   
        mode      3.000000  2.000000  2.000000         3.000000  3.000000  4.000000   
        NAs       4.000000  5.000000  5.000000         7.000000  2.000000  3.000000   
        
                      Rock  Metal or Hardrock      Punk  Hiphop, Rap  Reggae, Ska  \textbackslash{}
        mean      3.761952           2.361470  2.456088     2.910537     2.769691   
        std       1.184861           1.372995  1.301105     1.375677     1.214434   
        min       1.000000           1.000000  1.000000     1.000000     1.000000   
        25\%       3.000000           1.000000  1.000000     2.000000     2.000000   
        50\%       4.000000           2.000000  2.000000     3.000000     3.000000   
        75\%       5.000000           3.000000  3.000000     4.000000     4.000000   
        max       5.000000           5.000000  5.000000     5.000000     5.000000   
        skewness -0.702586           0.604915  0.441427     0.037217     0.156497   
        kurtosis -0.419187          -0.934732 -0.959379    -1.250059    -0.900509   
        mode      5.000000           1.000000  1.000000     4.000000     3.000000   
        NAs       6.000000           3.000000  8.000000     4.000000     7.000000   
        
                  Swing, Jazz  Rock n roll  Alternative    Latino  Techno, Trance  
        mean         2.759960     3.141575     2.828514  2.842315        2.338983  
        std          1.257936     1.237269     1.347173  1.327902        1.324099  
        min          1.000000     1.000000     1.000000  1.000000        1.000000  
        25\%          2.000000     2.000000     2.000000  2.000000        1.000000  
        50\%          3.000000     3.000000     3.000000  3.000000        2.000000  
        75\%          4.000000     4.000000     4.000000  4.000000        3.000000  
        max          5.000000     5.000000     5.000000  5.000000        5.000000  
        skewness     0.146457    -0.108936     0.162211  0.188489        0.569644  
        kurtosis    -0.997739    -0.917436    -1.129404 -1.099347       -0.906037  
        mode         3.000000     3.000000     3.000000  2.000000        1.000000  
        NAs          6.000000     7.000000     7.000000  8.000000        7.000000  
\end{Verbatim}


            
    \begin{Verbatim}[commandchars=\\\{\}]
{\color{incolor}In [{\color{incolor}15}]:} \PY{k+kn}{import} \PY{n+nn}{seaborn} \PY{k}{as} \PY{n+nn}{sbn}
         \PY{k+kn}{import} \PY{n+nn}{matplotlib}\PY{n+nn}{.}\PY{n+nn}{pyplot} \PY{k}{as} \PY{n+nn}{plt}
         \PY{n}{corr} \PY{o}{=} \PY{n}{music}\PY{o}{.}\PY{n}{corr}\PY{p}{(}\PY{p}{)} \PY{c+c1}{\PYZsh{}попарная корреляция Пиросона }
         \PY{n}{sbn}\PY{o}{.}\PY{n}{heatmap}\PY{p}{(}\PY{n}{corr}\PY{p}{,} 
                     \PY{n}{xticklabels}\PY{o}{=}\PY{n}{corr}\PY{o}{.}\PY{n}{columns}\PY{o}{.}\PY{n}{values}\PY{p}{,}
                     \PY{n}{yticklabels}\PY{o}{=}\PY{n}{corr}\PY{o}{.}\PY{n}{columns}\PY{o}{.}\PY{n}{values}\PY{p}{,}
                     \PY{n}{vmax}\PY{o}{=}\PY{o}{.}\PY{l+m+mi}{8}\PY{p}{,} \PY{n}{square}\PY{o}{=}\PY{k+kc}{True}\PY{p}{)}
         \PY{n}{plt}\PY{o}{.}\PY{n}{show}\PY{p}{(}\PY{p}{)}
\end{Verbatim}


    \begin{center}
    \adjustimage{max size={0.9\linewidth}{0.9\paperheight}}{output_2_0.png}
    \end{center}
    { \hspace*{\fill} \\}
    
    \begin{Verbatim}[commandchars=\\\{\}]
{\color{incolor}In [{\color{incolor}69}]:} \PY{k+kn}{import} \PY{n+nn}{numpy} \PY{k}{as} \PY{n+nn}{np}
         \PY{k+kn}{import} \PY{n+nn}{matplotlib}
         \PY{k+kn}{import} \PY{n+nn}{matplotlib}\PY{n+nn}{.}\PY{n+nn}{pyplot} \PY{k}{as} \PY{n+nn}{plt}
         \PY{k+kn}{from} \PY{n+nn}{sklearn}\PY{n+nn}{.}\PY{n+nn}{model\PYZus{}selection} \PY{k}{import} \PY{n}{cross\PYZus{}val\PYZus{}score}
         
         \PY{n}{n\PYZus{}features} \PY{o}{=} \PY{n+nb}{len}\PY{p}{(}\PY{n}{music}\PY{o}{.}\PY{n}{columns}\PY{p}{)}
         \PY{n}{n\PYZus{}components} \PY{o}{=} \PY{n}{np}\PY{o}{.}\PY{n}{arange}\PY{p}{(}\PY{l+m+mi}{0}\PY{p}{,} \PY{l+m+mi}{4}\PY{p}{)}
         \PY{n}{fa\PYZus{}scores} \PY{o}{=} \PY{p}{[}\PY{p}{]}
         \PY{n}{fa} \PY{o}{=} \PY{n}{FactorAnalysis}\PY{p}{(}\PY{p}{)}
         \PY{n}{fa}\PY{o}{.}\PY{n}{fit}\PY{p}{(}\PY{n}{music}\PY{p}{)}
         \PY{k}{for} \PY{n}{n} \PY{o+ow}{in} \PY{n}{n\PYZus{}components}\PY{p}{:}
             \PY{n}{fa}\PY{o}{.}\PY{n}{n\PYZus{}components} \PY{o}{=} \PY{n}{n}
             \PY{n}{fa\PYZus{}scores}\PY{o}{.}\PY{n}{append}\PY{p}{(}\PY{n}{np}\PY{o}{.}\PY{n}{mean}\PY{p}{(}\PY{n}{cross\PYZus{}val\PYZus{}score}\PY{p}{(}\PY{n}{fa}\PY{p}{,} \PY{n}{music}\PY{p}{)}\PY{p}{)}\PY{p}{)}
         \PY{n}{n\PYZus{}components\PYZus{}fa} \PY{o}{=} \PY{n}{n\PYZus{}components}\PY{p}{[}\PY{n}{np}\PY{o}{.}\PY{n}{argmax}\PY{p}{(}\PY{n}{fa\PYZus{}scores}\PY{p}{)}\PY{p}{]}
         \PY{n+nb}{print}\PY{p}{(}\PY{n}{n\PYZus{}components\PYZus{}fa}\PY{p}{)}
\end{Verbatim}


    \begin{Verbatim}[commandchars=\\\{\}]
3

    \end{Verbatim}

    \begin{Verbatim}[commandchars=\\\{\}]
{\color{incolor}In [{\color{incolor}72}]:} \PY{k+kn}{from} \PY{n+nn}{sklearn}\PY{n+nn}{.}\PY{n+nn}{decomposition} \PY{k}{import} \PY{n}{FactorAnalysis}
         \PY{n}{factor} \PY{o}{=} \PY{n}{FactorAnalysis}\PY{p}{(}\PY{n}{n\PYZus{}components}\PY{o}{=}\PY{l+m+mi}{3}\PY{p}{)}
         \PY{n}{factor}\PY{o}{.}\PY{n}{fit}\PY{p}{(}\PY{n}{music}\PY{p}{)}
         \PY{n+nb}{print} \PY{p}{(}\PY{p}{(}\PY{n}{pd}\PY{o}{.}\PY{n}{DataFrame}\PY{p}{(}\PY{n}{factor}\PY{o}{.}\PY{n}{components\PYZus{}}\PY{p}{,}\PY{n}{columns}\PY{o}{=}\PY{n}{music}\PY{o}{.}\PY{n}{columns}\PY{p}{)}\PY{p}{)}\PY{o}{.}\PY{n}{transpose}\PY{p}{(}\PY{p}{)}\PY{p}{)}
\end{Verbatim}


    \begin{Verbatim}[commandchars=\\\{\}]
                          0         1         2
Dance             -0.303149  0.665696 -0.425720
Folk               0.311782  0.402950  0.319220
Country            0.313558  0.329835  0.181988
Classical music    0.553627  0.341860  0.451575
Musical            0.262464  0.568232  0.329080
Pop               -0.271719  0.560852 -0.217120
Rock               0.801508 -0.061497 -0.227127
Metal or Hardrock  0.876026 -0.348435 -0.235428
Punk               0.816995 -0.247060 -0.472734
Hiphop, Rap       -0.374550  0.465859 -0.570731
Reggae, Ska        0.342663  0.340363 -0.349031
Swing, Jazz        0.604564  0.515972  0.191347
Rock n roll        0.780514  0.306515 -0.001735
Alternative        0.737268 -0.003769 -0.025753
Latino             0.043111  0.813650  0.124829
Techno, Trance    -0.234808  0.353415 -0.452851

    \end{Verbatim}

    \begin{Verbatim}[commandchars=\\\{\}]
{\color{incolor}In [{\color{incolor} }]:} 
\end{Verbatim}



    % Add a bibliography block to the postdoc
    
    
    
    \end{document}
